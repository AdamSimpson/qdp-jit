\documentclass[12pt,letterpaper]{article}


% Somewhat wider and taller page than in art12.sty
\topmargin -0.4in  \headsep 0.4in  \textheight 9.0in
\oddsidemargin 0.25in  \evensidemargin 0.25in  \textwidth 6in

\footnotesep 14pt
\floatsep 28pt plus 2pt minus 4pt      % Nominal is double what is in art12.sty
\textfloatsep 40pt plus 2pt minus 4pt
\intextsep 28pt plus 4pt minus 4pt

\begin{document}
\pagestyle{empty} 
\begin{center}
{\bfseries\large QDP++}
\end{center}
\bigskip

\date{Sept. 25, 2002}

\noindent
{\bf\large Description:}

QDP++ is a C++ data-parallel interface for Lattice field theory
applications. Many of the issues have been discussed in the
corresponding C interface description which can be found at
{\tt http://www.jlab.org/\~edwards/qcdapi/Level2API\_latest.pdf}. The
motivation for this development comes from the U.S. Dept. of Energy
SciDAC program.

The QDP interface provides an environment somewhat similar to 
Fortran 90 - namely data-parallel operations (operator/infix form)
which can be applied on lattice wide objects. The interface provides a
level of abstraction such that high-level user code written using the
API can be run unchanged on a single processor workstation or a
collection of multiprocessor nodes with parallel communications.
Architectural dependencies are hidden below the interface. A variety
of types for the site elements are provided. To achieve good
performance, overlapping communication and computation primitives are
provided.

\medskip

\noindent
{\em Supported operations:}

All QDP types are of type QDPType. Operations that are supported are below.
Convention: protoyypes are basically of the form:

\begin{verbatim}
QDPType<T1>  unary_function(const QDPType<T>&)
QDPType<T1>  binary_function(const QDPType<T>&, const QDPType<T>&)
\end{verbatim}

\medskip


\noindent
{\bf\large Infix operators:}

\noindent
{\em Unary infix (e.g., "operator-"):}

\begin{verbatim}
- : negation
+ : unaryplus
~ : bitwise not
! : boolean not
\end{verbatim}

\noindent
{\em Binary infix (e.g., "operator+"):}

\begin{verbatim}
+  : addition
-  : subtraction
*  : multiplication
/  : division
%  : mod
&  : bitwise and
|  : bitwise or
^  : bitwise exclusive or
<< : left-shift
>> : right-shift
\end{verbatim}

\noindent
{\em Comparisons (returning booleans, e.g., "operator<"):}

\begin{verbatim}
<, <=, >, >=, ==, !=
&& : and of 2 booleans
|| : or of 2 boolean
\end{verbatim}


\noindent
{\em Assignments (e.g., "operator+="):}

\begin{verbatim}
=, +=, -=, *=, /=, %=, |=, &=, ^=, <<=, >>=
\end{verbatim}

\noindent
{\em Trinary:}

\begin{verbatim}
where : the C trinary "?" operator
\end{verbatim}

\medskip


\noindent
{\bf\large Functions (standard C math lib):}

\noindent
{\em Unary:}

\begin{verbatim}
cos, sin, tan, acos, asin, atan, cosh, sinh, tanh,
exp, log, log10, sqrt,
ceil, floor, fabs
\end{verbatim}

\noindent
{\em Binary:}

\begin{verbatim}
ldexp, pow, fmod, atan2
\end{verbatim}

\medskip


\noindent
{\bf\large Additional functions (specific to QDP):}

\noindent
{\em Unary:}

\begin{verbatim}
conj             : hermitian conjugate (adjoint)
trace            : matrix trace
real             : real part
imag             : imaginary part
colorTrace       : trace over color indices
spinTrace        : trace over spin indices
noColorTrace     : trace over all but color indices
noSpinTrace      : trace over all but spin indices
multiplyI        : multiplies argument by imag "i"
multiplyMinusI   : multiplies argument by imag "-i"
localNorm2       : on fibers computes trace(conj(source)*source)
\end{verbatim}


\noindent
{\em Binary:}

\begin{verbatim}
cmplx              : returns complex object   arg1 + i*arg2
localInnerproduct  : at each site computes trace(conj(arg1)*arg2)
\end{verbatim}

\medskip


\noindent
{\bf\large Global reductions:}

\begin{verbatim}
sum(arg1)               : sum over lattice indices returning 
                          object of same fiber type
norm2(arg1)             : sum(localNorm2(arg1))
innerproduct(arg1,arg2) : sum(localInnerproduct(arg1,arg2))
\end{verbatim}


\medskip

\noindent
{\bf\large Other functions:}

\begin{verbatim}
spinProject(QDPType psi, int dir, int isign)
		    : applies spin projection  (1 + isign*gamma_\mu)*psi
		      returning a half spin vector or matrix

spinReconstruct(QDPType psi, int dir, int isign)
		    : applies spin reconstruction from  (1 + isign*gamma_\mu)*psi
		      returning a full spin vector or matrix

su2_extract(QDPType r_0, r_1, r_2, r_3, int index, QDPType source)
		    : Extract components r_k proportional to SU(2) submatrix su2_index
		      from the "SU(Nc)" matrix V. The SU(2) matrix is parametrized in the
		      sigma matrix basis. su2_index = 0, ..., Nc*(Nc-1)/2
		      r_0,r_1,r_2,r_3 <- source(su2_index)  [SU(N) field]

sun_fill(QDPType dest, int index, QDPType r_0, r_1, r_2, r_3)
		    : Fill an SU(Nc) matrix V with the SU(2) submatrix su2_index
		      paramtrized by r_k in the sigma matrix basis.
		      dest(su2_index) <- r_0,r_1,r_2,r_3
\end{verbatim}

\bigskip


\noindent
{\bf\large Types:}

Mathematically, QDP types follow a fiber-bundle terminology.  There is
a base space (a Nd dimensional lattice) with fibers at each site.  The
fiber type describes data primitives on each site.  Lattice fields are
defined and always allocated over the entire lattice; however, the
operations can be narrowed to only a subset of sites.  The primitive
types at each site are represented as the (tensor) product space of,
for example, a vector space over color components with a vector space
over spin components and complex valued elements.

Generically objects transform under different spaces with a tensor
product structure like

\begin{verbatim}
                         Color      Spin
Gauge fields:   Product(Matrix(Nc),Scalar)
Fermions:       Product(Vector(Nc),Vector(Ns))
Scalars:        Product(Scalar,    Scalar)
Propagators:    Product(Matrix(Nc),Matrix(Ns))
Gamma:          Product(Scalar,    Matrix(Ns))
\end{verbatim}

\noindent
Nd is the number of space-time dimensions
\noindent
Nc is the dimension of the color vector space
\noindent
Ns is the dimension of the spin vector space
\noindent
NOTE: these parameters are compile time defined in  qdp++/params.h

E.g., gauge fields can left-multiply fermions via color matrix by color
vector but is diagonal in spin space (spin scalar times spin vector).
A gamma matrix can right-multiply a propagator (spin matrix times
spin matrix) but is diagonal in color space (color matrix times color
scalar).

Types in the QDP interface are parameterized by a variety of types including:
\begin{itemize}
\item {\em Word type}: 
  int, float, double, Boolean . Basic machine types.
\item {\em Reality type}: 
  complex or scalar. This is where the idea of a complex number
  lives.
\item {\em Primitive type}: 
  scalar, vector, matrix, etc. This where the concept of a gauge or
  spin field lives. There can be many more types here.
\item {\em Inner grid type}: 
  scalar or lattice. Supports vector style architectures.
\item {\em Outer grid type}:
  OScalar or OLattice. Supports super-scalar style architectures. In
  combination with Inner grid can support a mixed mode like a
  super-scalar architecture with short length vector instructions.
\end{itemize}

There are templates classes for each of the type variants listed
above. The interface relies heavily on templates for
composition. There is very little inheritance. The basic objects are
constructed (at the users choice) by compositions like the following:

\begin{verbatim}
typedef OLattice<Pscalar<ColorMatrix<Complex<Real>, Nc> > > LatticeGauge
typedef OLattice<SpinVector<ColorVector<Complex<Real>, Nc> > > LatticeFermion
\end{verbatim}

The ordering of types here is suitable for a microprocessor architecture.

\end{document}

