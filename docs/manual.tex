% -*-LaTeX-*- document QDP API (C++ Language Binding) version 1.4.13
%
\documentclass[12pt,letterpaper]{article}

% Somewhat wider and taller page than in art12.sty
\topmargin -0.4in  \headsep 0.4in  \textheight 9.0in
\oddsidemargin 0.25in  \evensidemargin 0.25in  \textwidth 6in

\footnotesep 14pt
\floatsep 28pt plus 2pt minus 4pt      % Nominal is double what is in art12.sty
\textfloatsep 40pt plus 2pt minus 4pt
\intextsep 28pt plus 4pt minus 4pt

\newcommand{\cpp}{C++}

%%%%%%%%%%%%%%%%%%%%%%%%%%%%%%%%%%%%%%%%%%%%%%%%%%%%%%%%%%%%%%%%%%%%%%
% Math ops
%%%%%%%%%%%%%%%%%%%%%%%%%%%%%%%%%%%%%%%%%%%%%%%%%%%%%%%%%%%%%%%%%%%%%%
\renewcommand{\Re}{\mathop{\rm Re}}
\renewcommand{\Im}{\mathop{\rm Im}}
\newcommand{\integer}{\mathop{\rm int}}
\newcommand{\Tr}{\mathop{\rm Tr}}
\newcommand{\diag}{\mathop{\rm diag}}
\newcommand{\op}{\mathop{\rm op}}
\renewcommand{\not}{\mathop{\rm not}}


%%%%%%%%%%%%%%%%%%%%%%%%%%%%%%%%%%%%%%%%%%%%%%%%%%%%%%%%%%%%%%%%%%%%%%
% Names of types
%%%%%%%%%%%%%%%%%%%%%%%%%%%%%%%%%%%%%%%%%%%%%%%%%%%%%%%%%%%%%%%%%%%%%%
\newcommand{\tConstant}{constant}

\newcommand{\tLatticeReal}{LatticeReal}
\newcommand{\tLatticeComplex}{LatticeComplex}
\newcommand{\tLatticeInt}{LatticeInt}
\newcommand{\tLatticeBoolean}{LatticeBoolean}
\newcommand{\tLatticeColorMatrix}{LatticeColorMatrix}
\newcommand{\tLatticeSpinMatrix}{LatticeSpinMatrix}
\newcommand{\tLatticeHalfFermion}{LatticeHalfFermion}
\newcommand{\tLatticeDiracFermion}{LatticeDiracFermion}
\newcommand{\tLatticeStaggeredFermion}{LatticeStaggeredFermion}
\newcommand{\tLatticeDiracPropagator}{LatticeDiracPropagator}
\newcommand{\tLatticeStaggeredPropagator}{LatticeStaggeredPropagator}
\newcommand{\tLatticeRandomState}{LatticeSeed}

\newcommand{\tReal}{Real}
\newcommand{\tComplex}{Complex}
\newcommand{\tInt}{Integer}
\newcommand{\tBoolean}{Boolean}
\newcommand{\tColorMatrix}{ColorMatrix}
\newcommand{\tHalfFermion}{HalfFermion}
\newcommand{\tDiracFermion}{DiracFermion}
\newcommand{\tStaggeredFermion}{StaggeredFermion}
\newcommand{\tDiracPropagator}{DiracPropagator}
\newcommand{\tStaggeredPropagator}{StaggeredPropagator}
\newcommand{\tRandomState}{Seed}

% Aliases/typedefs
\newcommand{\tSpinMatrix}{SpinMatrix}
\newcommand{\tSpinVector}{SpinVector}
\newcommand{\tColorVector}{ColorVector}
\newcommand{\tLatticeColorVector}{LatticeColorVector}
\newcommand{\tLatticeSpinVector}{LatticeSpinVector}

% Fixed precision
\newcommand{\tRealD}{RealD}
\newcommand{\tLatticeRealD}{LatticeRealD}
\newcommand{\tLatticeComplexD}{LatticeComplexD}
\newcommand{\tLatticeColorMatrixD}{LatticeColorMatrixD}
\newcommand{\tLatticeHalfFermionD}{LatticeHalfFermionD}
\newcommand{\tLatticeDiracFermionD}{LatticeDiracFermionD}
\newcommand{\tLatticeStaggeredFermionD}{LatticeStaggeredFermionD}

%%%%%%%%%%%%%%%%%%%%%%%%%%%%%%%%%%%%%%%%%%%%%%%%%%%%%%%%%%%%%%%%%%%%%%
% Prototype elements
%%%%%%%%%%%%%%%%%%%%%%%%%%%%%%%%%%%%%%%%%%%%%%%%%%%%%%%%%%%%%%%%%%%%%%
\newcommand{\namespace}{QDP}
\newcommand{\allTypes}{{\tt R, C, I, G, H, D, V, P, M, S}}
\newcommand{\allNumericTypes}{All numeric types}
\newcommand{\allFloatTypes}{All floating types}
\newcommand{\simLatticeVariants}{Also corresponding lattice variants}
\newcommand{\ttdash}{{::}}
\newcommand{\itt}{\it Type}
\newcommand{\extraarg}{}

%%%%%%%%%%%%%%%%%%%%%%%%%%%%%%%%%%%%%%%%%%%%%%%%%%%%%%%%%%%%%%%%%%%%%%
% Patterns for prototypes
%%%%%%%%%%%%%%%%%%%%%%%%%%%%%%%%%%%%%%%%%%%%%%%%%%%%%%%%%%%%%%%%%%%%%%
\newcommand{\protoUnary}{{\tt {\it Type func}(const {\it Type}\& a)}}
\newcommand{\protoUnaryShift}{{\tt {\it Type} shift(const {\it Type}\& a, int sign, int dir)}}
\newcommand{\protoUnaryQual}[1]{{\tt {\it Type} #1(const {\it Type}\& a)}}
\newcommand{\protoBinaryQual}[1]{{\tt {\it Type} #1(const {\it Type1}\& a, const {\it Type2}\& b)}}
\newcommand{\protoBBinaryQual}[1]{{\tt {\it Type} #1(const {\it Type}\& a, const {\it Type}\& b)}}
\newcommand{\protoBoolBinaryQual}[1]{{\tt {\it Type2} #1(const {\it Type}\& a, const {\it Type1}\& b)}}
\newcommand{\protoAssignmentQual}[1]{{\tt {\it Type} #1(const {\it Type}\& r, const {\it Type}\& a)}}


%%%%%%%%%%%%%%%%%%%%%%%%%%%%%%%%%%%%%%%%%%%%%%%%%%%%%%%%%%%%%%%%%%%%%%
% QDP macro names
%%%%%%%%%%%%%%%%%%%%%%%%%%%%%%%%%%%%%%%%%%%%%%%%%%%%%%%%%%%%%%%%%%%%%%
\newcommand{\qdpNcMacro}{{\tt \namespace}\ttdash{\tt Nc }}
\newcommand{\qdpPrecisionMacro}{{\tt \namespace}\ttdash{\tt Precision}}
\newcommand{\qdpColorsMacro}{{\tt \namespace}\ttdash{\tt Colors}}

%%%%%%%%%%%%%%%%%%%%%%%%%%%%%%%%%%%%%%%%%%%%%%%%%%%%%%%%%%%%%%%%%%%%%

\begin{document}
\title{
   QDP++ Data Parallel Interface for QCD  \\{\large Version 1.4.13}
}
\author{Robert G. Edwards\\
SciDAC Software Coordinating Committee}

\maketitle
%%%%%%%%%%%%%%%%%%%%%%%%%%%%%%%%%%%%%%%%%%%%%%%%%%%%%%%%%%%%%%%%%%%%%%
\section{Introduction}
%%%%%%%%%%%%%%%%%%%%%%%%%%%%%%%%%%%%%%%%%%%%%%%%%%%%%%%%%%%%%%%%%%%%%%
This is a user's guide for the \cpp{} binding for the QDP Data Parallel
Applications Programmer Interface developed under the auspices of the
U.S. Department of Energy Scientific Discovery through Advanced
Computing (SciDAC) program.

The QDP Level 2 API has the following features:
\begin{itemize}
\item Provides data parallel operations (logically SIMD) on all sites
across the lattice or subsets of these sites.
\item Operates on lattice objects, which have an
implementation-dependent data layout that is not visible above this
API.
\item Hides details of how the implementation maps onto a given
architecture, namely how the logical problem grid (i.e. lattice) is
mapped onto the machine architecture.
\item Allows asynchronous (non-blocking) shifts of lattice level
objects over any permutation map of sites onto sites. However, from
the user's view these instructions appear blocking and in fact may be
so in some implementation.
\item Provides broadcast operations (filling a lattice quantity from a
scalar value(s)), global reduction operations, and lattice-wide
operations on various data-type primitives, such as matrices, vectors,
and tensor products of matrices (propagators).
\item
Operator syntax that support complex expression constructions.
\end{itemize}

%%%%%%%%%%%%%%%%%%%%%%%%%%%%%%%%%%%%%%%%%%%%%%%%%%%%%%%%%%%%%%%%%%%%%%
\section{Datatypes}
%%%%%%%%%%%%%%%%%%%%%%%%%%%%%%%%%%%%%%%%%%%%%%%%%%%%%%%%%%%%%%%%%%%%%%
\label{sec.datatypes}

The $N_d$ dimensional lattice consists of all the space-time sites in
the problem space.  Lattice data are fields on these sites.  A data
primitive describes data on a single site.  The lattice fields consist
of the primitives over all sites.  We do not define data types
restricted to a subset of the lattice --- rather, lattice fields occupy
the entire lattice.
The primitive types at
each site are represented as the (tensor) product space of, for
example, a vector space over color components with a vector space over
spin components and complex valued elements.

\subsection{Type Structure}
\label{sec:typestructure}

Generically objects transform under different spaces with a tensor
product structure as shown below:

\begin{displaymath}
\begin{array}{llcccccl}
                  & {\it Lattice} &    &  {\it Color} &     & {\it Spin}  &     & {\it Complexity}\\
{\rm Gauge\ fields:}& {\tt Lattice} &\otimes& {\tt Matrix(Nc)}&\otimes& {\tt Scalar}    &\otimes& {\tt Complex}\\
{\rm Fermions:}    & {\tt Lattice} &\otimes& {\tt Vector(Nc)}&\otimes& {\tt Vector(Ns)}&\otimes& {\tt Complex}\\
{\rm Scalars:}     & {\tt Scalar}  &\otimes& {\tt Scalar}    &\otimes& {\tt Scalar}    &\otimes& {\tt Scalar} \\
{\rm Propagators:} & {\tt Lattice} &\otimes& {\tt Matrix(Nc)}&\otimes& {\tt Matrix(Ns)}&\otimes& {\tt Complex}\\
{\rm Gamma:}       & {\tt Scalar}  &\otimes& {\tt Scalar}    &\otimes& {\tt Matrix(Ns)}&\otimes& {\tt Complex}\\
\end{array}
\end{displaymath}

\noindent
\verb|Nd| is the number of space-time dimensions\\
\noindent
\verb|Nc| is the dimension of the color vector space\\
\noindent
\verb|Ns| is the dimension of the spin vector space\\

Gauge fields can left-multiply fermions via color matrix times color
vector but is diagonal in spin space (spin scalar times spin vector).
A gamma matrix can right-multiply a propagator (spin matrix times
spin matrix) but is diagonal in color space (color matrix times color
scalar).

Types in the QDP interface are parameterized by a variety of types including:
\begin{itemize}
\item {\em Word type}: 
  int, float, double, bool. Basic machine types.
\item {\em Reality type}: 
  complex or scalar. This is where the idea of a complex number
  lives.
\item {\em Primitive type}: 
  scalar, vector, matrix, etc. This is where the concept of a gauge or
  spin field lives. There can be many more types here.
\item {\em Inner grid type}: 
  scalar or lattice. Supports vector style architectures.
\item {\em Outer grid type}:
  scalar or lattice. Supports super-scalar style architectures. In
  combination with Inner grid can support a mixed mode like a
  super-scalar architecture with short length vector instructions.
\end{itemize}

There are template classes for each of the type variants listed
above. The interface relies heavily on templates for
composition - there is very little inheritance. The basic objects are
constructed (at the users choice) by compositions like the following:

\begin{small}
\begin{verbatim}
typedef OLattice<PScalar<PColorMatrix<RComplex<float>, Nc> > > LatticeColorMatrix
typedef OLattice<PSpinVector<PColorVector<RComplex<float>, Nc>, Ns> > LatticeFermion
\end{verbatim}
\end{small}
%
The classes PScalar, PSpinVector, PColorMatrix,
PColorVector are all subtypes of a primitive type. The relative
ordering of the classes is important. It is simply a user convention
that spin is used as the second index (second level of type
composition) and color is the third. The ordering of types can be
changed. From looking at the types one can immediately decide what
operations among objects makes sense.


\subsection{Generic Names}

The linear algebra portion of the QDP API is designed to resemble the
functionality that is available in the Level 1 QLA API and the C Level
QDP API.  Thus the datatypes and function naming conventions are
similar.  Predefined names for some generic lattice field datatypes
are listed in the table below. Because the API is based heavily on
templates, the possible types allowed is much larger than listed
below.
\begin{center}
\begin{tabular}{|l|l|}
\hline
name                       & description \\
\hline
{\tt \tLatticeReal               } & real \\
{\tt \tLatticeComplex            } & complex \\
{\tt \tLatticeInt                } & integer \\
{\tt \tLatticeColorMatrix              } & $N_c \times N_c$ complex matrix \\
{\tt \tLatticeHalfFermion        } & two-spin, $N_c$ color spinor \\
{\tt \tLatticeDiracFermion       } & four-spin, $N_c$ color spinor \\
{\tt \tLatticeStaggeredFermion   } & one-spin,  $N_c$ color spinor \\
{\tt \tLatticeDiracPropagator    } & $4N_c \times 4N_c$ complex matrix \\
{\tt \tLatticeStaggeredPropagator} & $N_c \times N_c$ complex matrix \\
{\tt \tLatticeRandomState        } & implementation dependent \\
\hline
\end{tabular}
\end{center}
%
Single site (lattice wide constant fields) versions of types exist
without the {\tt Lattice} preprended.  All types and operations
defined for QDP live within a \cpp{} namespace called QDP thus
ensuring no type conflicts with other namespaces.

\subsection{Specific Types for Color and Precision}

According to the chosen color and precision, names for specific
floating point types are constructed from names for generic types.
Thus {\tt \tLatticeColorMatrix} becomes {\tt LatticeColorMatrix}{\it PC},
where the precision {\it P} is {\tt D} or {\tt F} according to the
table below
\begin{center}
\begin{tabular}{|l|l|}
\hline
abbreviation & description \\
\hline
{\tt D}            & double precision \\
{\tt F}            & single precision \\
\hline
\end{tabular}
\end{center}
and {\it C} is {\tt 2}, {\tt 3}, or some arbitrary {\tt N}, if color is a
consideration. Note, the value of {\tt N} is an arbitrary compile
time constant.

If the datatype carries no color, the color label is omitted. Also,
if the number of color components is the same as the compile
time constant, then the color label can be omitted.
Integers also have no precision label. The specification of
precision and number of colors is not needed for functions because
of overloading.

For example, the type
%
\begin{verbatim}
   LatticeDiracFermionF3
\end{verbatim}
%
describes a lattice quantity of single-precision four-spin, 
three-color spinor field.

\subsection{Color and Precision Uniformity}

The only place that the number of color or spin components occur
is through instance of the global constant variables $N_c$, and $N_s$. These 
are only directly used in the typedef constructions of user defined types. Nothing
restricts a user from constructing types for other number of colors. In fact,
the use of $N_c$ in the construction of user defined types is simply a
convenience for the user, and as such a user can use any integer that is
reasonable. The API merely requires that the types used in operations are conforming.

However, in standard coding practice it is assumed that a user keeps
one of the precision, color, and spin options in force throughout the
compilation.  So as a rule all functions in the interface take
operands of the same precision, color, and number of spin
components. As with data type names, function names come in generic
color-, spin- and precision-specific forms, as described in the next
section.  Exceptions to this rule are functions that explicitly
convert from double to single precision and vice versa.  If the user
choose to adopt color and precision uniformity, then all variables can
be defined with generic types and all functions accessed through
generic names.  The prevailing color is defined through
the compile time constant $N_c$.  The interface automatically
translates data type names and function names to the appropriate
specific type names through typedefs.  With such a scheme and careful
coding, changing only the compile time $N_c$ and the QDP library
converts code from one color and precision choice to another.

\subsection{Breaking Color and Precision Uniformity}

It is permissible for a user to mix precision and color choices.  This
is done by declaring variables with specific type names, using
functions with specific names, and making appropriate precision
conversions when needed. 

\newpage 
%%%%%%%%%%%%%%%%%%%%%%%%%%%%%%%%%%%%%%%%%%%%%%%%%%%%%%%%%%%%%%%%%%%%%%
\section{QDP Functions}
%%%%%%%%%%%%%%%%%%%%%%%%%%%%%%%%%%%%%%%%%%%%%%%%%%%%%%%%%%%%%%%%%%%%%%

The QDP functions are grouped into the following categories:
\begin{enumerate}
\item Entry and exit from QDP
\item Layout utilities
\item Data parallel functions
\item Data management utilities
\item Subset definition
\item Shift creation
\item I/O utilities
\item Temporary exit and reentry
\end{enumerate}

\subsection{Entry and exit from QDP}
\label{sec:init_qdp}

QDP must be initialized before any other routine can be used.
The initialization is broken into two steps -- initializing
the underlying hardware and initializing the layout.
%
\paragraph{Initialization of QDP}

\begin{flushleft}
  \begin{tabular}{|l|l|}
  \hline
  Prototype      & \verb|void QDP_initialize(int *argc, char ***argv)|\\
    \hline
  Purpose        & Places the hardware into a known state. \\
\hline
  Example  & \verb|QDP_initialize();| \\
   \hline
 \end{tabular}
\end{flushleft}
%
This routine will be responsible for initializing any hardware
like the physical layer of the message passing system. 
For compatibility with QMP, the addresses of the main programs \verb|argc|
and \verb|argv| must be passed. They may be modified.


\paragraph{Shutdown of QDP}

\begin{flushleft}
  \begin{tabular}{|l|l|}
  \hline
  Prototype      & \verb|void QDP_finalize()|\\
    \hline
  Purpose        & Shutdown QDP. \\
\hline
  Example  & \verb|QDP_finalize();| \\
   \hline
 \end{tabular}
\end{flushleft}
%
This call provides for an orderly shutdown of QDP.  It is called by
all nodes. It concludes all communications, does housekeeping, if
needed and performs a barrier wait for all nodes.  Then it returns
control to the calling process.

\paragraph{Panic exit from QDP}

\begin{flushleft}
  \begin{tabular}{|l|l|}
  \hline
  Prototype      & \verb|void QDP_abort(int status)|\\
    \hline
  Purpose        & Panic shutdown of the process. \\
\hline
  Example  & \verb|QDP_abort(1);| \\
   \hline
 \end{tabular}
\end{flushleft}
%
This routine may be called by one or more nodes.  It sends kill
signals to all nodes and exits with exit status \verb|status|.

\paragraph{Entry into QDP}

\begin{flushleft}
  \begin{tabular}{|l|l|}
  \hline
  Prototype      & \verb|void Layout::create()|\\
    \hline
  Purpose        & Starts QDP with layout parameters in \verb|Layout|. \\
\hline
  Example  & \verb|Layout::create();| \\
   \hline
 \end{tabular}
\end{flushleft}
%
The routine \verb|Layout::create()| is called once by all nodes and
starts QDP operations. It calls the layout routine with the
parameters set in the namespace \verb|Layout| specifying the layout.
The layout is discussed in Section~\ref{sec:layout}.

This step is separated from the \verb|QDP_initialize()| above so
layout parameters can be read and broadcasted to the nodes.  Otherwise
the layout parameters have to be set from the environment or fixed in
the compilation.


\paragraph{Exit from QDP}

\begin{flushleft}
  \begin{tabular}{|l|l|}
  \hline
  Prototype      & \verb|void Layout::destroy()|\\
    \hline
  Purpose        & Exits QDP. \\
\hline
  Example  & \verb|Layout::destroy();| \\
   \hline
 \end{tabular}
\end{flushleft}
%
This call provides for an orderly exit from QDP.  It is called by all
nodes. It concludes all communications, does housekeeping, if needed
and performs a barrier wait for all nodes. The communication layer is
not finalized.

\subsection{Layout utilities}
\label{sec:layout}

Routines for constructing the layout are collected in the namespace
\verb|Layout|. The {\it set}ter and {\it get}ter routines provide
a way to set parameters like the lattice size.

The layout creation function determines which nodes get which lattice
sites and in what linear order the sites are stored. The \verb|Layout|
namespace has entry points that allow a user to inquire about the
lattice layout to facilitate accessing single site data from a QDP
lattice field. For code written entirely with other QDP calls, these
routines may be ignored by the user, with the exception of the useful
routine \verb|latticeCoordinate|.  However, if a user removes data
from a QDP lattice object (see \verb|expose| or \verb|extract|) and
wishes to manipulate the data on a site-by-site basis, the global
entry points provided here are needed to locate the site data.

Some implementations may have a built-in tightly constrained layout.
In flexible implementations there may be several layout choices,
thereby allowing the user the freedom to select one that works best
with a given application.  Furthermore, such implementations may allow
the user to create a custom layout to replace one of the standard
layouts.  As long as the custom layout procedure provides the entry
points and functionality described here, compatibility with the
remainder of the QDP library is assured.

\subsubsection{QDP setup}

\paragraph{Layout creation}
\mbox{}

\noindent
The layout creation routine \verb|Layout::create()| defined in
Section~\ref{sec:init_qdp} generates predefined lattice subsets for
specifying even, odd, and global subsets of the lattice. The \verb|rb|
set can be dereferenced to produce the \verb|even| and \verb|odd| subsets:
%
\begin{verbatim}
Subset even, odd, all,  rb[0], rb[1]
\end{verbatim}
%
It also creates the nearest-neighbor shifts for each coordinate
direction.

\paragraph{Defining the layout}
\mbox{}

\noindent
There are set/accessor functions to specify the lattice geometry used
in the layout. Generically, the accessors have the form:

\begin{flushleft}
  \begin{tabular}{|l|l|}
  \hline
  Generic      & \verb|void Layout::set<something>(<param>)|\\
  \hline
  Purpose      & Set one of the site data layout configurations. \\
\hline
  Example  & \verb|Layout::setLattSize(size);| \\
   \hline
 \end{tabular}
\end{flushleft}

The type of input information needed by the layout is as follows:
\begin{enumerate}
\item	Number of dimensions $N_d$. Must be the compile time dimensions.
\item	Lattice size  (e.g., $L_0$, $L_1$, ..., $L_{N_d-1}$)
\item	SMP flag
%\item	Number of nodes and processors per node
%\item Rank of the current node or processor
\end{enumerate}

These parameters are accessed and set with the following functions:
\begin{flushleft}
  \begin{tabular}{|l|l|}
  \hline
  Generic      & \verb|void Layout::setLattSize(const multi1d<int>& size)|\\
  \hline
  Purpose      & Set the lattice size for the data layout. \\
  \hline
  Default      & No default value. Must always be set. \\
\hline
  Example  & \verb|Layout::setLattSize(size);| \\
   \hline
 \end{tabular}
\end{flushleft}

\begin{flushleft}
  \begin{tabular}{|l|l|}
  \hline
  Generic      & \verb|void Layout::setSMPFlag(bool)|\\
  \hline
  Purpose      & Turn on using multi-processor/threading \\
  \hline
  Default      & Default value is false - single thread of execution. \\
  \hline
  Example  & \verb|Layout::setSMPFlag(true);| \\
   \hline
 \end{tabular}
\end{flushleft}

\begin{flushleft}
  \begin{tabular}{|l|l|}
  \hline
  Generic      & \verb|void Layout::setNumProc(int N)|\\
  \hline
  Purpose      & In a multi-threaded implementation, use \verb|N| processors. \\
  \hline
  Default      & Default value is 1 - single thread of execution. \\
  \hline
  Example  & \verb|Layout::setNumProc(2);| \\
   \hline
 \end{tabular}
\end{flushleft}

\subsubsection{Generic layout information}
        
The following global entry points are provided in the \verb|Layout| namespace.
They provide generic user information.

\paragraph{Returning the spacetime coordinates}

\begin{flushleft}
  \begin{tabular}{|l|l|}
  \hline
  Prototype      & {\tt \tLatticeInt} \verb|Layout::latticeCoordinate(int d)|\\
  \hline
  Purpose        & The {\tt d}th spacetime coordinate. \\
  \hline
  Example        & {\tt \tLatticeInt{} coord = Layout::latticeCoordinate(2);} \\
  \hline
 \end{tabular}
\end{flushleft}
%
The call \verb|Layout::latticeCoordinate(d)| returns an integer lattice field with a
value on each site equal to the integer value of the \verb|d|th
space-time coordinate on that site.

\paragraph{Lattice volume}

\begin{flushleft}
  \begin{tabular}{|l|l|}
  \hline
  Prototype      & {\tt int} \verb|Layout::vol()|\\
  \hline
  Purpose        & Return the total lattice volume\\
  \hline
  Example        & {\tt int vol = Layout::vol();} \\
  \hline
 \end{tabular}
\end{flushleft}

\subsubsection{Entry points specific to the layout}
        
The additional global entry points are provided in the \verb|Layout| namespace.
They reveal some information specific to the implementation.

\paragraph{Node number of site}

\begin{flushleft}
  \begin{tabular}{|l|l|}
  \hline
  Prototype      & \verb|int Layout::nodeNumber(const multi1d<int>& x)|\\
    \hline
  Purpose        & Returns logical node number containing site \verb|x|. \\
\hline
  Example  & \verb|node = Layout::nodeNumber(x);| \\
   \hline
 \end{tabular}
\end{flushleft}

\paragraph{Linear index of site}

\begin{flushleft}
  \begin{tabular}{|l|l|}
  \hline
  Prototype      & \verb|int Layout::linearSiteIndex(const multi1d<int>& x)|\\
    \hline
  Purpose        & Returns the linearized index for the lattice site \verb|x|. \\
\hline
  Example  & \verb|int k = Layout::linearSiteIndex(x);| \\
   \hline
 \end{tabular}
\end{flushleft}

\paragraph{Map node and linear index to coordinate}

\begin{flushleft}
  \begin{tabular}{|l|l|}
  \hline
  Prototype      & \verb|multi1d<int> Layout::siteCoords(int node, int index)|\\
    \hline
  Purpose        & Returns site coordinate \verb|x| for the given node \\
                 & \verb|node| and linear index \verb|index|. \\
\hline
  Example  & \verb|multi1d<int> lc = Layout::siteCoords(n, i);| \\
   \hline
 \end{tabular}
\end{flushleft}

\paragraph{Number of sites on a node}

\begin{flushleft}
  \begin{tabular}{|l|l|}
  \hline
  Prototype      & \verb|int Layout::sitesOnNode()|\\
  \hline
  Purpose        & Returns number of sites assigned to a node.\\
  \hline
  Example        & \verb|int num = Layout::sitesOnNode();| \\
  \hline
 \end{tabular}
\end{flushleft}
%
The linear index returned by \verb|Layout::linearSiteIndex()| 
ranges from 0 to \verb|Layout:sitesOnNode()|$ - 1$.

\newpage

\subsection{Data Parallel Functions}

Data parallel functions are described in detail in
Sec.~\ref{sec:supported}.  
In the \cpp{} API, there are overloaded functions that can be applied
to site or lattice wide objects. Arbitrarily complicated expressions
can be built from these functions.  The design of the API describes
that all operations are to be performed site-wise. The only connection
between sites is via a map or shift function.

The class of operations are generically described by site-wise
operations (the ``linear algebra'' part of the API), and shift (or
map) versions. The latter generically involves communications among
processors in a parallel implementation.

The operator style provided by the API thus allows operations like
the following:
%
\begin{verbatim}
  Lattice A, B;
  LatticeColorMatrix U;
  B = U * A;
\end{verbatim}
%
From the type declarations
%
\begin{small}
\begin{verbatim}
typedef OLattice<PScalar<PColorMatrix<RComplex<float>, Nc> > > LatticeColorMatrix
typedef OLattice<PSpinVector<PColorVector<RComplex<float>, Nc>, Ns> > LatticeFermion
\end{verbatim}
\end{small}
%
one can see a {\tt OLattice} multiplies a {\tt OLattice}. At each
site, the {\tt U} field is a scalar in spin space, thus a {\tt
PScalar} multiplies a {\tt PSpinVector} - a vector in spin space. For
each spin component, there is a {\tt PColorMatrix} multipling a {\tt
PColorVector}. The multplications involve complex numbers. 

Thus we see that mathematically the expression carries out the product
%
\begin{displaymath}
{\rm B}_\alpha^{i}(x) = {\rm U}^{i j}(x) * {\rm A}_\alpha^{j}(x)
\end{displaymath}
%
for all lattice coordinates {\tt x} belonging to the subset {\tt all}.
Here {\tt A} and {\tt B} are objects of lattice Dirac fermion fields
and {\tt U} is an onject of type lattice gauge field. The superscripts
$i$, $j$ refer to the color indices and the subscript $\alpha$ refers
to the spin index. For each spin and color component, the
multiplication is over complex types.

This tensor product factorization of types allows for potentially a
huge variety of mathematical objects. The operations between the objects
is determined by their tensor product structure.

% As described in Section~\ref{sec:typestructure}

The API allows for operations to be narrowed to a subset of sites.
The infix notation does not allow for extra arguments to be passed to
an operation, so the subset is fixed via the target.  
The API mandates that there is in use in even a complex 
operation, namely the target specifies the subset to use. 
To narrow an operation to a specific subset, one
specifies the subset in the target as follows:
%
\begin{verbatim}
  chi[even] = u * psi;
\end{verbatim}
%
which will store the result of the multiplication on only the 
{\em even} subset.

The \cpp{} API differs from the C API signficantly in the name of
functions. In \cpp{}, there is no need for naming conventions for
the functions since one can overload the function name on the types
of its arguments. More significantly, the C API uses a functional
style where the destination of an operation is part of the arguments
for an operation, and all functions return void. The \cpp{} API uses
an operator/infix style allowing complex expressions to be built.


\subsubsection{Constant Arguments}

In some cases it is desirable to keep an argument constant over the
entire subset.  For example the function
%
\begin{verbatim}
   Complex z;
   LatticeFermion c, b;
   c[s] = z * b;
\end{verbatim}
%
multiplies a lattice field of color vectors by a complex constant as in
%
\begin{verbatim}
   c[x] = z*b[x]
\end{verbatim}
%
for {\tt x} in subset {\tt s}.

\subsubsection{Functions}

In the \cpp{} API all operations are functions that act on their
argument and most functions return their results. Except for explicit
shift functions and global reductions, these functions are point-wise.
The \cpp{} API differs from the C API in that there are no combined
operations like adjoint with a multiply.  Instead, one simply calls
the adjoint function.  Thus
%
\begin{verbatim}
  c = adj(u)*b
\end{verbatim}
%
carries out the product
%
\begin{verbatim}
  c[x] = adj(u[x])*b[x]
\end{verbatim}
%
for all sites {\tt x} in subset {\tt all}.

\subsubsection{Gamma matrices}

Multiplication of spin vectors and matrices by $\gamma$-matrices is provided.
%
\begin{verbatim}
  LatticeDiracFermion c, b;
  int n;
  c = Gamma(n) * b;
\end{verbatim}
%
Where the multiplciation is spin matrix times a spin vector. Right multiplication
is also supported.
\begin{verbatim}
  LatticePropagator q, r;
  int n;
  q = r * Gamma(n);
\end{verbatim}

The \verb|Gamma(n)| provides an enumeration of all possible
$\gamma$-matrix combinations.  See the Section~\ref{sec:spin} for more
details.

\subsubsection{Shift}

A shift function is a type of map that maps sites from one lattice
site to another. In general, maps can be permutation maps but there
are nearest neighbor shift functions provided by default.
See the discussion of shifts below
in Section~\ref{sec:shifts}.
Thus
%
\begin{verbatim}
  c[s] = shift(b,sign,dir)
\end{verbatim}
%
shifts an object along the direction specified by {\tt dir} and 
{\tt sign} for all sites {\tt x} in destination subset {\tt s}.


\subsection{Creating and destroying lattice fields}

The declaration of an object of some type say {\tt LatticeReal} will
call a constructor. The implementation guarantees the object is fully
created and all memory needed for it is allocated.  Thus, there is no
need for the user to use {\tt new} to create an object. The use of
pointers is discouraged.  When an object goes out of scope, a
destructor is called which will guarantee all memory associated with
the object is released.  

There is no aliasing or referencing of two objects with the same
internal data storage. Each object a user can construct has its own
unique storage.

\subsection{Array container objects}
\label{sec:arrays}

For convenience, the API provides array container classes with much
limited facility compared to the Standard Template Library. In
particular, one, two, three, and four dimensional array container
classes are available. The benefit of two and higher dimension classes
is that they can be allocated after they are declared. This is in
contrast to the STL technique, which builds multi-dimensional arrays
out of nested one-dimensional array, and one must allocate a nested
array of array classes by looping over the individual elements
allocating each one.

An array of container classes is constructed as follows:
%
\begin{verbatim}
multi1d<LatticeComplex> r(Nd); // a 1-D array of LatticeComplex
multi2d<Real> foo(2,3);        // a 2-D array of Real with first index slowest
\end{verbatim}


\subsection{Function objects}
\label{sec:funcobj}

Function objects are used in the constructions of Sets/subsets and
maps/shifts. The objects created by maps are themselves function
objects.  They serve the role as functions, but because of their class
structure can also carry state.

A function object has a struct/class declaration. The key part is
the function call operator. A generic declaration is something like:
%
\begin{verbatim}
struct MyFunction
{
  MyFunction(int dir) : mu(dir) {}
  Real operator()(const int& x)
    {\* operates on x using state held in mu and returns a Real *\}

  int mu;
}
\end{verbatim}

A user can then use an object of type MyFunction like a function:
%
\begin{verbatim}
MyFunction  foo(37); // hold 37 within foo
int x;
Real boo = foo(x);  // applies foo via operator()
\end{verbatim}


\subsection{Subsets}

It is sometimes convenient to partition the lattice into multiple
disjoint subsets (e.g. time slices or checkerboards).  Such subsets
are defined through a user-supplied function that returns a range of
integers $0,1,2,\ldots{},n-1$, so that if $f(x) = i$, then site $x$ is
in partition $i$.  A single subset may also be defined by limiting the
range of return values to a single value (i.e. 0).  This procedure may
be called more than once, and sites may be assigned to more than one
subset.  Thus, for example an even site may also be assigned to a time
slice subset and one of the subsets in a 32-level checkerboard scheme.
A subset definition remains valid until is destructor is called.

\paragraph{Defining a set}

Subsets are first defined through the construction of an object
of type \verb|Set| using a function object. This function object is
a derived type of \verb|SetFunc|. Function objects are described
in Section~\ref{sec:funcobj}. Subsets are defined through the 
data type \verb|Subset|.

\begin{flushleft}
\begin{tabular}{|l|p{5.0in}|}
\hline
Prototype    & \verb|Set::make(const SetFunc& func)|\\
             & \verb|int SetFunc::operator()(const multi1d<int>& x)|\\
             & \verb|int SetFunc::numSubsets()|\\
\hline
Purpose      & Creates a Set that holds \verb|numSubsets| subsets based on \verb|func|.\\
\hline
Requirements & The \verb|func| is a derived type of \verb|SetFunc| and
               maps lattice coordinates to a partition number.\\
             & The function in \verb|func.numSubsets()| returns number of partitions.\\
\hline
Example  & \verb|Set timeslice|; \\
         & \verb|class timesliceFunc : public SetFunc;| \\
         & \verb|timeslice.make(timesliceFunc);| \\
   \hline
  \end{tabular}
\end{flushleft}
%
Here is an explicit example for a timeslice:
%
\begin{verbatim}
struct TimeSliceFunc : public SetFunc
{
  TimeSliceFunc(int dir): mu(dir) {}

  // Simply return the mu'th coordinate
  int operator()(const multi1d<int>& coord)
  {return coord[mu];}

  // The number of subsets is the length of the lattice
  // in direction mu
  int numSubsets() {return Layout::lattSize()[mu];}

  int mu; // state
}

Set timeslice;
timeslice.make(TimeSliceFunc(3)) // makes timeslice in direction 3
\end{verbatim}

It is permissible to call \verb|Set.make()| with a function object
having only 1 subset.  In this case the partition function must return
zero if the site is in the subset and nonzero if not.  (Note, this is
opposite the ``true'', ``false'' convention in C).

\paragraph{Extracting a subset}

A subset is returned from indexing a \verb|Set| object.

\begin{flushleft}
  \begin{tabular}{|l|l|}
  \hline
  Prototype    & \verb|Subset Set::operator[](int i)| \\
    \hline
Purpose        & Returns the \verb|i|-th subset from a \verb|Set| object.\\
    \hline
Example  & \verb|Set timeslice|; \\
         & \verb|Subset origin = timeslice[0];| \\
   \hline
  \end{tabular}
\end{flushleft}

The \verb|Set::make()| functions allocates all memory associated with
a \verb|Set|. A \verb|Subset| holds a reference info to the original
\verb|Set|. A destructor call on a \verb|Set| frees all memory.

\paragraph{Using a subset}

A subset can be used in an assignment to restrict sites involved in
a computation:
%
\begin{verbatim}
  LatticeComplex r, a, b;
  Subset s;
  r[s] = 17 * a * b;
\end{verbatim}
%
will multiply {\tt 17 * a * b} onto {\tt r} only on sites in the 
subset {\tt s}.

\subsection{Maps and shifts}
\label{sec:shifts}

Shifts are general communication operations specified by any
permutation of sites.  Nearest neighbor shifts are a special case.
Thus, for example,
%
\begin{verbatim}
  LatticeHalfFermion a, r;
  r[s] = shift(a,sign,dir);
\end{verbatim}
%
shifts the half fermion field {\tt a} along direction {\tt dir},
forward or backward according to {\tt sign}, placing the result in the
field {\tt r}.  Nearest neighbor shifts are specified by values of
{\tt dir} in the range $[0,N_d-1]$.  The sign is $+1$ for shifts from
the positive direction, and $-1$ for shifts from the negative
direction.  That is, for {\tt sign}$= +1$ and {\tt dir}$= \mu$, $r(x)
= a(x+\hat \mu)$. For more general permutations, {\tt dir} is missing
and {\tt sign} specifies the permutation or its inverse.

The subset restriction applies to the destination field {\tt r}.  Thus
a nearest neighbor shift operation specifying the even subset shifts
odd site values from the source {\tt a} and places them on even site
values on the destination field {\tt r}.

\paragraph{Creating shifts for arbitrary permutations}

The user must first create a function object for use in the
map creation as described in Section~\ref{sec:funcobj}. 
Thus to use the make a map one uses a function object in the map
creation:

\begin{flushleft}
\begin{tabular}{|l|p{4.0in}|}
\hline
Prototype    & \verb|Map::make(const MapFunc& func)|\\
\hline
Purpose      & Creates a map specified by the permutation map function 
               object {\tt func}. \\
\hline
Requirements & The \verb|func| is a derived type of \verb|MapFunc| and must
               have a \verb|multi1d<int> operator()(const multi1d<int>& d)| 
               member function that maps a source site to \verb|d|. \\
\hline
Result       & Creates an object of type map which has a function call 
               \verb|template<class T> T Map::operator()(const T& a)| \\
\hline
Example      & \verb|Map naik;|\\
             & \verb|LatticeReal r,a;| \\
             & \verb|r = naik(a);| \\
\hline
  \end{tabular}
\end{flushleft}
%
The coordinate
map function object {\tt func} above that is handed to the
map creation function \verb|Map::make()| maps lattice coordinates
of the the destination to the source lattice coordinates.
After construction, the function object of type \verb|Map| can be
used like any function via the \verb|operator()|. It can be applied
to all QDP objects in an expression.

The function object has an operator that given
a coordinate will return the source site coordinates. An example
is as follows:
\begin{verbatim}
struct naikfunc : public MapFunc
{
  naik(int dir) : mu(dir) {}
  multi1d<int> operator()(const multi1d<int>& x)
    {\* maps x to x + 3*mu  where mu is direction vector *\}

  int mu;
}
\end{verbatim}
%

For convenience, there are predefined Map functions named \verb|shift|
that can shift by 1 unit backwards or forwards in any lattice direction.
They have the form
%
\begin{verbatim}
shift(const QDPType& source, int sign, int dir);
\end{verbatim}

The construction of a \verb|Map| object allocates all the necessary
memory needed for a shift. Similarly, a destructor call on a \verb|Map|
object frees memory.

\subsection{Temporary entry and exit from QDP}

For a variety of reasons it may be necessary to remove data from QDP
structures.  Conversely, it may be necessary to reinsert data into QDP
structures.  For example, a highly optimized linear solver may
operate outside QDP.  The operands would need to be extracted from QDP
fields and the eventual solution reinserted.  It may also be useful to
suspend QDP communications temporarily to gain separate access to the
communications layer.  For this purpose function calls are provided to
put the QDP implementation and/or QDP objects into a known state,
extract values, and reinsert them.

%% \paragraph{Exposing QDP data}
%% 
%% \indent
%% {\bf THIS ROUTINE IS NOT IMPLEMENTED. IT IS A LOW LEVEL AND HIGHLY PLATFORM SPECIFIC OPTIMIZATION.}
%% 
%% \begin{flushleft}
%% \begin{tabular}{|l|l|}
%% \hline
%% Prototype    & {\tt {\it QLA\_Type *} QDP\_expose}({\tt {\it Type}\& src})\\
%% \hline
%% Purpose        & Deliver data values from field {\tt src}. \\
%% \hline
%% \itt     & \allNumericTypes \\
%% \hline
%% Example  & \verb|r = QDP_expose(a);| \\
%% \hline
%% \end{tabular}
%% \end{flushleft}
%% %
%% This function grants direct access to the data values contained in the
%% QDP field {\tt src}.  The return value is a pointer to an array of QLA
%% data {\tt dest} of type {\it QLA\_Type}.  The order of the data is given by
%% \verb|Layout::linearSiteIndex|.  No QDP operations except \verb|QDP_insert| are
%% permitted on exposed data until \verb|QDP_reset| is called. (See
%% below.)

%% \paragraph{Returning control of QDP data}
%% 
%% {\bf\large THIS ROUTINE IS NOT IMPLEMENTED.}
%% 
%% \begin{flushleft}
%% \begin{tabular}{|l|l|}
%% \hline
%% Prototype      & \verb|void QDP_reset|({\tt {\it Type}\& field})\\
%% \hline
%% Purpose        & Returns control of data values to QDP. \\
%% \hline
%% \itt     & \allNumericTypes \\
%% \hline
%% Example  & \verb|QDP_reset(r);| \\
%% \hline
%% \end{tabular}
%% \end{flushleft}
%% %
%% This call signals to QDP that the user is ready to resume QDP
%% operations with the data in the specified field.

\paragraph{Extracting QDP data}

\begin{flushleft}
\begin{tabular}{|l|l|}
  \hline
  Prototype    & \verb|void QDP_extract|({\tt multi1d<{\it Type2}>\& dest}, 
                   {\tt const {\it Type1}\& src}, \\
               & \qquad{\tt const Subset\& s})\\
\hline
Purpose        & Copy data values from field {\tt src} to array {\tt dest}. \\
\hline
  {\it Type1} & All lattice types \\
  {\it Type2} & All corresponding scalar lattice types \\
\hline
Example  & \verb|LatticeFermion a;| \\
         & \verb|multi1d<Fermion> r(Layout::sitesOnNode());| \\
         & \verb|QDP_extract(r,a,even);| \\
\hline
  \end{tabular}
\end{flushleft}
%
The user must allocate the space of size \verb|Layout::sitesOnNode()| for the 
destination array before calling this function, regardless of the size of
the subset.

This function copies the data values contained in the QDP field {\tt
src} to the destination field.  Only values belonging to the specified
subset are copied.  Any values in the destination array not associated
with the subset are left unmodified.  The order of the data is given
by \verb|Layout::linearSiteIndex|.  Since a copy is made, QDP operations
involving the source field may proceed without disruption.

\paragraph{Inserting QDP data}

\begin{flushleft}
\begin{tabular}{|l|l|}
\hline
Prototype      & \verb|void QDP_insert|({\tt {\it Type1}\& dest}, 
                 {\tt const multi1d<{\it Type2}>\& src}, \\
               & \qquad{\tt const Subset\& s})\\
\hline
Purpose        & Inserts data values from array {\tt src}. \\
\hline
  {\it Type1} & All lattice types \\
  {\it Type2} & All corresponding scalar lattice types \\
\hline
Example  & \verb|multi1d<Fermion> a(Layout::sitesOnNode());|\\
         & \verb|LatticeFermion r;| \\
         & \verb|QDP_insert(r,a,odd);| \\
\hline
  \end{tabular}
\end{flushleft}
%
Only data associated with the specified subset are inserted.  Other
values are unmodified.  The data site order must conform to
\verb|Layout::linearSiteIndex|.  This call, analogous to a fill operation, is
permitted at any time and does not interfere with QDP operations.

\paragraph{Suspending QDP communications}

If a user wishes to suspend QDP communications temporarily and carry
on communications by other means, it is first necessary to call
\verb|QDP_suspend|.

\begin{flushleft}
\begin{tabular}{|l|l|}
  \hline
  Prototype      & \verb|void QDP_suspend(void)| \\
\hline
Purpose        & Suspends QDP communications. \\
\hline
Example  & \verb|QDP_suspend();| \\
\hline
  \end{tabular}
\end{flushleft}
%
No QDP shifts can then be initiated until \verb|QDP_resume| is called.
However QDP linear algebra operations without shifts may proceed.

\paragraph{Resuming QDP communications}

To resume QDP communications one uses
\begin{flushleft}
\begin{tabular}{|l|l|}
  \hline
  Prototype      & \verb|void QDP_resume(void)| \\
\hline
Purpose        & Restores QDP communications. \\
\hline
Example  & \verb|QDP_resume();| \\
\hline
  \end{tabular}
\end{flushleft}

\newpage

%%%%%%%%%%%%%%%%%%%%%%%%%%%%%%%%%%%%%%%%%%%%%%%%%%%%%%%%%%%%%%%%%%%%%%
\section{Simple I/O utilities}
\label{sec:simpleio}
%%%%%%%%%%%%%%%%%%%%%%%%%%%%%%%%%%%%%%%%%%%%%%%%%%%%%%%%%%%%%%%%%%%%%%

\subsection{Basic structure}

There are three main types of user accessible classes for {\em simple}
file I/O - Text, XML and Binary. For each of these classes there is a
Reader and a Writer version. Each support I/O for {\em any}\  QDP
defined scalar and lattice quantity as well as the standard \cpp\
builtin types like \verb|int| and \verb|float|. These classes all
read/write to one primary node in the computer, namely
\verb|Layout::primaryNode()| or node 0. Lattice quantities are read/written
lexicographically as one contiguous field with the first index in the
lattice size \verb|Layout::lattSize()| varying the fastest.

A record structure format is available to store both metadata and binary data.
The metadata uses the XML format. The binary I/O functions support more advanced
I/O mechanisms and is described in Section~\ref{sec:qio}.

The \cpp\ standard IO streams \verb|cout|, \verb|cerr| and \verb|cin|
are, of course, provided by the language but will not work as
expected. Namely, the output functions will write on all nodes, and
\verb|cin| will try to read from all nodes and fail. QDP defined
fields \verb|QDP_cout|, \verb|QDP_cerr|, and \verb|QDP_cin| are
provided as replacements, and will write/read from only the primary
node. In addition, a setup function is provided that will permanently
change \verb|cout|, \verb|cerr| and \verb|cin| to have the same
semantics (primary node usage) as their QDP counterparts. 
{\bf Implementation note: these std IO stream replacements are not yet
available in QDP++}.

\subsection{Text Reading and Writing}
\label{sec:textio}

\paragraph{TextReader member functions and global functions}

\begin{flushleft}
  \begin{tabular}{|l|l|}
  \hline
  Open to read   & \verb|TextReader::TextReader(const string& filename)|\\
                 & \verb|void TextReader::open(const string& filename)|\\
  \hline
  Close          & \verb|TextReader::~TextReader()|\\
                 & \verb|void TextReader::close()|\\
  \hline
  Open?          & \verb|bool TextReader::is_open()| \\
  \hline
  Any IO errors? & \verb|bool TextReader::fail()| \\
  \hline
  Input a type \verb|T| & \verb|TextReader& operator>>(TextReader&, T&)| \\
  \hline
 \end{tabular}
\end{flushleft}

\paragraph{TextWriter member functions and global functions}

\begin{flushleft}
  \begin{tabular}{|l|l|}
  \hline
  Open to write  & \verb|TextWriter::TextWriter(const string& filename)|\\
                 & \verb|void TextWriter::open(const string& filename)|\\
  \hline
  Close          & \verb|TextWriter::~TextWriter()|\\
                 & \verb|void TextWriter::close()|\\
  \hline
  Open?          & \verb|bool TextWriter::is_open()| \\
  \hline
  Any IO errors? & \verb|bool TextWriter::fail()| \\
  \hline
  Output a type \verb|T| & \verb|TextWriter& operator<<(TextWriter&, const T&)| \\
  \hline
 \end{tabular}
\end{flushleft}
%
To read and write ascii text from the file, use the standard operators
familiar in the \cpp\ Standard Library. E.g.,
%
\begin{small}
\begin{verbatim}
TextWriter out("foo");
Real a = 1.2;
Complex b = cmplx(Real(-1.1), Real(2.2));
out << a << endl << b << endl;
out.close();

TextReader in("foo");
Real a;
Complex b;
in >> a >> b;
in.close();
\end{verbatim}
\end{small}
%
The \verb|TextWriter| functions would produce a file ``foo'' that looks
like
\begin{verbatim}
1.2
-1.1 2.2
\end{verbatim}

\subsection{XML Reading and Writing}
\label{sec:xmlio}

XML is intended as the standard format for user produced human
readable data as well as metadata. The XML format is always a tree of
key/value pairs with arbitrarily deep nesting. Here, the keys are
variable names. The semantics imposed by QDP is that no key can be
repeated twice at the same nesting level. Also, the values are considered
one of three types -- a simple type, a structure, or an array of one of these
three types including array.

The path specification for the XML reader functions is XPath. Namely,
QDP only {\em requires} a simple unix like path to reference a
tag. With a simple path and nested reading, all data can be read from
a document.  However, more complicated queries are possible
allowing the user to read individual pieces of a document - e.g., a
single array element. See the XPath language specification for
additional information: http://www.w3.org/TR/xpath.html .

Further details on the document format of various types is given in
Section~\ref{sec:xmldoc}.

\paragraph{XMLReader member functions and global functions}

\begin{flushleft}
  \begin{tabular}{|l|l|}
  \hline
  Read file      & \verb|XMLReader::XMLReader(const string& filename)|\\
                 & \verb|void XMLReader::open(const string& filename)|\\
  \hline
  Read stream    & \verb|XMLReader::XMLReader(std::istream&)|\\
                 & \verb|void XMLReader::open(std::istream&)|\\
  \hline
  Read buffer    & \verb|XMLReader::XMLReader(const XMLBufferWriter&)|\\
                 & \verb|void XMLReader::open(const XMLBufferWriter&)|\\
  \hline
  Close          & \verb|XMLReader::~XMLReader()|\\
                 & \verb|void XMLReader::close()|\\
  \hline
  Open?          & \verb|bool XMLReader::is_open()| \\
  \hline
  Any IO errors? & \verb|bool XMLReader::fail()| \\
  \hline
  Input a type \verb|T| & \verb|void read(XMLReader&, const string& path, T&)| \\
  \hline
 \end{tabular}
\end{flushleft}
%
An example of reading a file is
\begin{verbatim}
XMLReader xml_in("foo");
int a;
Complex b;
multi1d<Real> c;
multi1d<Complex> d;
read(xml_in, "/bar/a", a);  // primitive type reader
read(xml_in, "/bar/b", b);  // QDP defined struct reader
read(xml_in, "/bar/c", d);  // array of primitive type reader
read(xml_in, "/bar/d", d);  // calls read(XMLReader, string, Complex) 
\end{verbatim}
%
The file ``foo'' might look like the following:
%
\begin{verbatim}
<?xml version="1.0"?>
<bar>
 <!-- A simple primitive type -->
 <a>17</d>

 <!-- Considered a structure -->
 <b>
   <re>1.0</re>
   <im>2.0</im>
 </b>

 <!-- A length 3 array of primitive types -->
 <c>1 5.6 7.2</c>

 <!-- A length 2 array of non-simple types -->
 <d>
   <elem>
     <re>1.0</re>
     <im>2.0</im>
   </elem>
   <elem>
     <re>3.0</re>
     <im>4.0</im>
   </elem>
 </d>
</bar>
\end{verbatim}
%

The user can defined their own reader functions following the same
overloaded syntax as the predefined ones. This allows one to nest
readers.
%
\begin{verbatim}
struct MyStruct {int a; Real b;};

void read(XMLReader& xml_in, const string path&, MyStruct& input)
{
  read(xml_in, path + "/a", input.a);
  read(xml_in, path + "/b", input.b);
}

XMLReader xml_in;  // user should initialize here
multi1d<MyStruct> foo;  // array size will be allocated in array reader
read(xml_in, "/root", foo); // will call user defined read above for each element
\end{verbatim}

As stated before, the path specification for a read is actually an
XPath query. For example, the user can read only one array element in the file
``foo'' above via an XPath query:
%
\begin{verbatim}
XMLReader xml_in("foo");
Complex dd;
read(xml_in, "/bar/d/elem[2]", dd);  // read second array elem of d, e.g. d[1]
\end{verbatim}



\paragraph{XMLWriter base class global functions}

\begin{flushleft}
  \begin{tabular}{|l|l|}
  \hline
  Start a group  & \verb|void push(XMLWriter&, const string& name)|\\
  End a group    & \verb|void pop(XMLWriter&, const string& name)|\\
  \hline
  Output a \verb|T| & \verb|void write(XMLWriter&, const string& path, const T&)| \\
  Output  & \verb|void write(XMLWriter&, const string& path, const XMLBufferWriter&)| \\
          & \verb|XMLWriter& operator<<(XMLWriter&, const XMLBufferWriter&)| \\
          & \verb|void write(XMLWriter&, const string& path, const XMLReader&)| \\
          & \verb|XMLWriter& operator<<(XMLWriter&, const XMLReader&)| \\
  \hline
 \end{tabular}
\end{flushleft}
%
The \verb|XMLWriter| is an abstract base class for three concrete
classes which allow to write into a memory buffer, a file, or write an
array of objects in a series of steps.

\paragraph{XMLBufferWriter derived class member functions}

\begin{flushleft}
  \begin{tabular}{|l|l|}
  \hline
  Return entire buffer  & \verb|string XMLBufferWriter::str()|\\
  Return only root element & \verb|string XMLBufferWriter::printRoot()|\\
  \hline
 \end{tabular}
\end{flushleft}


\paragraph{XMLFileWriter derived class member functions}

\begin{flushleft}
  \begin{tabular}{|l|l|}
  \hline
  File to write  & \verb|XMLFileWriter::XMLFileWriter(const string& filename)|\\
                 & \verb|void XMLFileWriter::open(const string& filename)|\\
  \hline
  Close          & \verb|XMLFileWriter::~XMLFileWriter()|\\
                 & \verb|void XMLFileWriter::close()|\\
  \hline
  Open?          & \verb|bool XMLFileWriter::is_open()| \\
  \hline
  Any IO errors? & \verb|bool XMLFileWriter::fail()| \\
  \hline
  Flush          & \verb|void XMLFileWriter::flush()| \\
  \hline
 \end{tabular}
\end{flushleft}
%
Similar to the read case, the user can also create a tower of writer functions.
In addition, the user can create memory held buffered output that can be
used for metadata. Similarly, a user can go back and forth from readers to
writers.
%
\begin{verbatim}
XMLBufferWriter xml_buf;
push(xml_buf, "bar");
write(xml_buf, "a", 1);  // write /bar/a = 1
pop(xml_buf);

XMLReader xml_in(xml_buf);  // re-parse the xml_buf
int a;
read(xml_in, "/bar/a", a); // now have 1 in a

XMLFileWriter xml_out("foo");
xml_out << xml_in;     // will have ``bar'' as the root tag
xml_out.close();
\end{verbatim}


\paragraph{XMLArrayWriter derived class member and global functions}

\begin{flushleft}
  \begin{tabular}{|l|l|}
  \hline
  Constructor    & \verb|XMLArrayWriter::XMLArrayWriter(XMLWriter&, int size=-1)|\\
  \hline
  Close          & \verb|XMLArrayWriter::~XMLArrayWriter()|\\
                 & \verb|void XMLArrayWriter::close()|\\
  \hline
  Size           & \verb|int XMLArrayWriter::size()| \\
  \hline
  Start an array  & \verb|void push(XMLArrayWriter&)|\\
  End an array    & \verb|void pop(XMLArrayWriter&)|\\
  \hline
 \end{tabular}
\end{flushleft}
%
The Array class allows one to break writing an array into multiple pieces.
%
\begin{verbatim}
XMLFileWriter xml_out("foo");
XMLArrayWriter xml_array(xml_out, 350000);  // Note: a big array size here
push(xml_array, "the_name_of_my_array");
for(int i=0; i < xml_array.size(); ++i)
{
  push(xml_array);  // start next array element - name of tag already defined
  Real foo = i;
  write(xml_array, "foo", foo);
  pop(xml_array);   // finish this array element
}
\end{verbatim}


\subsection{XML document structure}
\label{sec:xmldoc}

QDP regards the structure of a document as composed of structures,
simple types, or arrays of structures or simple types. Simple types
are the usual builtin type of C and C++, namely \verb|int|,
\verb|float|, \verb|double|, \verb|bool|, etc. In addition, the QDP
scalar equivalents \verb|Integer|, \verb|Real|, \verb|Double|,
\verb|Boolean|, etc.  are also consider simple types. For instance, the
code snippet
%
\begin{verbatim}
int a = 3;
write(xml_out, "a", a);
\end{verbatim}
%
would produce
\begin{verbatim}
<a>3</a>
\end{verbatim}
%
indentities the name of a variable of a type with some values.
Following the XML Schema specifications, arrays of these simple types
have a simple form
\begin{verbatim}
<!-- produced from writing a multid<int> -->
<a>3 3 4 5</a>
\end{verbatim}

Again, following the XML Schema specifications all other objects are
considered complex (e.g., complicated) types. Hence, the document snippet
%
\begin{verbatim}
<?xml version="1.0"?>
<!-- Considered a structure of simple types, arrays and other structures -->
<bar>
 <!-- A simple primitive type -->
 <a>17</d>

 <!-- Considered a structure -->
 <b>
   <re>1.0</re>
   <im>2.0</im>
 </b>

 <!-- A length 3 array of primitive types -->
 <c>1 5.6 7.2</c>
\end{verbatim}
%
is viewed as a structure of other types:
\begin{verbatim}
struct bar_t
{
  int a;
  Complex b;
  multi1d<Real> c;
} bar;
\end{verbatim}
%
Hence, one views the \verb|push|/\verb|pop| semantics as a way of
dynamically constructing structures. 

\newpage
\paragraph{XML document format}

\begin{flushleft}
  \begin{tabular}{|l|l|}
  \hline
  \tInt,\tReal,\tRealD    & 
\mbox{\tt 
<a>3</a>
}
\\
\hline
  \tBoolean    & 
\mbox{\tt 
<a>yes</a>
}
\\
\hline
  string    & 
\mbox{\tt 
<a>hello world</a>
}
\\
\hline
  multi1d$<$int$>$    & 
\mbox{\tt 
<a>1 2 3</a>
}
\\
\hline
  \tComplex    & 
\mbox{\tt 
<a>
  <re>1.2</re>
  <im>2.0</im>
</a>
}
\\
\hline
  multi1d$<${\em Type}$>$    & 
\parbox{5in}{\tt 
<a>
  <elem>{\em Type}
  </elem>
  <elem>{\em Type}
  </elem>
</a>
}
\\
\hline
  multi1d$<$\tComplex$>$    & 
\parbox{5in}{\tt 
<a>
  <elem>
    <re>1.2</re>
    <im>2.0</im>
  </elem>
  <elem>
    <re>3</re>
    <im>5.0</im>
  </elem>
</a>
}
\\
\hline
  \tColorVector    & 
\parbox{5in}{\tt 
<a>
  <ColorVector>
    <elem row="0">
      <re>0</re>
      <im>1</im>
    </elem>
    <elem row="1">
      <re>2</re>
      <im>3</im>
    </elem>
  ...
  </ColorVector>
</a>
}
\\
\hline
  \tColorMatrix    & 
\parbox{5in}{\tt 
<a>
  <ColorMatrix>
    <elem row="0" col="0">
      <re>0</re>
      <im>1</im>
    </elem>
    <elem row="1" col="0">
      <re>2</re>
      <im>3</im>
    </elem>
  ...
  </ColorMatrix>
</a>
}
\\
\hline
  \tDiracPropagator    & 
\parbox{5in}{\tt 
<a>
  <SpinMatrix>
    <elem row="0" col="0">
      <ColorMatrix>
        <elem row="0" col="0">
          <re>0</re>
          <im>1</im>
        </elem>
  ...
      </ColorMatrix>
    </elem>
  </SpinMatrix>
</a>
}
\\
\hline
  Lattice{\em Type}    & 
\parbox{5in}{\tt 
<a>
  <OLattice>
    <elem site="0">{\em Type}
    </elem>
    <elem site="1">{\em Type}
    </elem>
  ...
  </OLattice>
</a>
}
\\
\hline
  \tLatticeReal    & 
\parbox{5in}{\tt 
<a>
  <OLattice>
    <elem site="0">1
    </elem>
    <elem site="1">2
    </elem>
  ...
  </OLattice>
</a>
}
\\
\hline
  \tLatticeColorVector    & 
\parbox{5in}{\tt 
<a>
  <OLattice>
    <elem site="0">
      <ColorVector>
        <elem row="0">
           <re>1</re>
           <im>2</im>
        </elem>
  ...
      </ColorVector>
    </elem>
  </OLattice>
</a>
}
\\

  \hline
 \end{tabular}
\end{flushleft}
%
A table of the document format for a variable ``a'' of various
types.

\newpage


\subsection{Binary Reading and Writing}
\label{sec:binaryio}

\paragraph{BinaryReader member functions and global functions}

\begin{flushleft}
  \begin{tabular}{|l|l|}
  \hline
  Open to read   & \verb|BinaryReader::BinaryReader(const string& filename)|\\
                 & \verb|void BinaryReader::open(const string& filename)|\\
  \hline
  Close          & \verb|BinaryReader::~BinaryReader()|\\
                 & \verb|void BinaryReader::close()|\\
  \hline
  Open?          & \verb|bool BinaryReader::is_open()| \\
  \hline
  Any IO errors? & \verb|bool BinaryReader::fail()| \\
  \hline
  Input a type \verb|T| & \verb|BinaryReader& operator>>(BinaryReader&, T&)| \\
                 & \verb|void read(BinaryReader&, T&)| \\
  \hline
 \end{tabular}
\end{flushleft}

\paragraph{BinaryWriter member functions and global functions}

\begin{flushleft}
  \begin{tabular}{|l|l|}
  \hline
  Open to write  & \verb|BinaryWriter::BinaryWriter(const string& filename)|\\
                 & \verb|void BinaryWriter::open(const string& filename)|\\
  \hline
  Close          & \verb|BinaryWriter::~BinaryWriter()|\\
                 & \verb|void BinaryWriter::close()|\\
  \hline
  Open?          & \verb|bool BinaryWriter::is_open()| \\
  \hline
  Any IO errors? & \verb|bool BinaryWriter::fail()| \\
  \hline
  Flush          & \verb|bool BinaryWriter::flush()| \\
  \hline
  Output a type \verb|T| & \verb|BinaryWriter& operator<<(BinaryWriter&, const T&)| \\
                 & \verb|void write(BinaryWriter&, const T&)| \\
  \hline
 \end{tabular}
\end{flushleft}
%
To read and write ascii text from the file, use the standard operators
familiar in the \cpp\ Standard Library. E.g.,
%
\begin{small}
\begin{verbatim}
BinaryWriter out("foo");
Real a;
LatticeColorMatrix b
write(out, a);
out << b;      // have choice of style
out.close();

BinaryReader in("foo");
Real a;
LatticeColorMatrix b;
in >> a >> b;
in.close();
\end{verbatim}
\end{small}

\newpage

%%%%%%%%%%%%%%%%%%%%%%%%%%%%%%%%%%%%%%%%%%%%%%%%%%%%%%%%%%%%%%%%%%%%%%
\section{QDP Record I/O utilities}
\label{sec:qio}
%%%%%%%%%%%%%%%%%%%%%%%%%%%%%%%%%%%%%%%%%%%%%%%%%%%%%%%%%%%%%%%%%%%%%%

%%%%%%%%%%%%%%%%%%%%%%%%%%%%%%%%%%%%%%%%%%%%%%%%%%%%%%%%%%%%%%%%%%%%%%
\subsection{Overview of File Format}
%%%%%%%%%%%%%%%%%%%%%%%%%%%%%%%%%%%%%%%%%%%%%%%%%%%%%%%%%%%%%%%%%%%%%%
\label{sec.fileformat}

\subsubsection{Binary QDP Files}

The binary file format has been designed with flexibility in mind. For
archiving purposes, the allowable file organization may be further
restricted.  Here we described the unrestricted format.

Binary QDP files are composed of a series of one or more application
records.  A single application record encodes a single QDP field or an
array of QDP fields of the same data type.  Physics metadata, managed
at the convenience of the applications programmer, is associated with
the file itself and with each application record as well. Above the
API the QDP file is viewed as follows:
%
\begin{itemize}
  \item File physics metadata
  \item Record 1 physics metadata
  \item Record 1 data
  \item Record 2 physics metadata
  \item Record 2 data
  \item etc.
\end{itemize}
%
For example, a file might record a series of staggered fermion
eigenvectors for a gauge field configuration.  Each record would map
to a single field of type \verb|LatticeColorVector|.  The file metadata
might include information about the gauge field configuration and the
record metadata might encode the eigenvalue and an index for the
eigenvector.

For another example, the gauge field configuration in four dimensions
is represented in QDP as an array of four color matrix fields.  The
configuration is conventionally written so that the four color matrices
associated with each site appear together.  A file containing a single
gauge field configuration would then consist of a single record
containing the array of four color matrices.

The API permits mixing records of different datatypes in the same
file.  While this practice may be convenient for managing projects, it
may be forbidden for archival files.

Additional metadata is automatically managed by QIO (without requiring
intervention by the applications programmer) to facilitate the
implementation and to check data integrity.  Thus the file actually
begins with QIO metadata and physics metadata and each application
record consists of five logical records.  Within QIO the file is
viewed as a series of logical records as follows:
%
\begin{itemize}
  \item File QIO metadata
  \item File physics metadata
  \item Record 1 QIO metadata
  \item Record 1 physics metadata
  \item Record 1 BinX description
  \item Record 1 binary data
  \item Record 1 checksum
  \item Record 2 QIO metadata
  \item Record 2 physics metadata
  \item Record 2 BinX description
  \item Record 2 binary data
  \item Record 2 checksum
  \item etc.
\end{itemize}
%

We have adopted the DIME standard (a generalization of the MIME
standard) for packaging the logical records.  File system limitations
may require splitting data into multiple physical files.  The
management of physical files is handled internally and is hidden from
the applications programmer.

\subsubsection{Multidump Files}

The API provides for rapid temporary writing of data to scratch disks
and reading from scratch disks.  In this case it is assumed that the
files are not intended for longer term storage.  Thus the file format
in this case is implementation-dependent and not standardized.  A
specific choice of format is described in the Appendix.

\subsubsection{Metadata Standard and Manipulation}

Metadata is required by a number of components in the I/O system. 
The XML I/O facilities described in Section~\ref{sec:xmlio} are used
for manipulating the metadata.

%%%%%%%%%%%%%%%%%%%%%%%%%%%%%%%%%%%%%%%%%%%%%%%%%%%%%%%%%%%%%%%%%%%%%%
\subsection{QDP/C++ Record API}
%%%%%%%%%%%%%%%%%%%%%%%%%%%%%%%%%%%%%%%%%%%%%%%%%%%%%%%%%%%%%%%%%%%%%%

As with standard Unix, a file must be opened before reading or
writing.  However, we distinguish file handles for both cases.  If the
system provides a parallel file system, it is possible for several
processors to read and write a single file. We call this mode
``parallel''.  Otherwise the file is read by a single processor and
the data delivered according to the distributed memory layout.  The
reverse occurs upon writing. We call this mode ``serial''. To allow
user choice where the architecture permits, we provide for requesting
either mode.  However, the request may be overridden if the system
permits only one mode.  Upon writing, we allow appending to an
existing file.

\paragraph{QDPRecordFileReader base class member functions and global functions}

\begin{flushleft}
  \begin{tabular}{|l|l|}
  \hline
  Open           & \verb|QDPRecordFileReader::QDPRecordFileReader(XMLReader& file_xml, |\\
                 & \qquad\verb|const string& path)|\\
                 & \verb|void QDPRecordFileReader::open(XMLReader& file_xml, |\\
                 & \qquad\verb|const string& path)|\\
  \hline
  Close          & \verb|QDPRecordFileReader::~QDPRecordFileReader()|\\
                 & \verb|void QDPRecordFileReader::close()|\\
  \hline
  Open?          & \verb|bool QDPRecordFileReader::is_open()| \\
  \hline
  EOF?           & \verb|bool QDPRecordFileReader::eof()| \\
  \hline
  Errors?        & \verb|bool QDPRecordFileReader::bad()| \\
  \hline
  Read a \verb|T| & \verb|void read(QDPRecordFileReader&, XMLReader& rec_xml, T&)| \\
  Array of \verb|T| & \verb|void vread(QDPRecordFileReader&, XMLReader& rec_xml, multi1d<T>&)| \\
  \hline
  Only xml & \verb|void peek(QDPRecordFileReader&, XMLReader& rec_xml)| \\
  \hline
  Next record & \verb|void skip(QDPRecordFileReader&)| \\
  \hline
 \end{tabular}
\end{flushleft}
%
Base class for the various derived classes. Here, \verb|vread| expects
the sites to be the slowest varying index and the array indices
(muli1d) inside of them. \verb|peek| returns only the metadata and
repositions back to the beginning of the record.  \verb|skip| skips to
the next logical record - it may read the data and discard
it. \verb|bad| states if any fatal errors have occurred. There are no
user functions that position within a logical record.

\paragraph{QDP record reader implementation classes}

\begin{flushleft}
  \begin{tabular}{|l|l|}
  \hline
  Single file read serially & \verb|QDPSingleFileSerialReader| \\
  Single file read in parallel & \verb|QDPSingleFileParallelReader| \\
  \hline
  Multi-dump file read serially & \verb|QDPMultiDumpSerialReader| \\
  Multi-dump file read in parallel & \verb|QDPMultiDumpParallelReader| \\
  \hline
 \end{tabular}
\end{flushleft}

{\bf Implementation note: QDP++ currently has only QDPSerialFileReader in place
of QDPSingleFileSerialReader}

\paragraph{QDPRecordFileWriter base class member functions and global functions}


\begin{flushleft}
  \begin{tabular}{|l|l|}
  \hline
  Open           & \verb|QDPRecordFileWriter::QDPRecordFileWriter(|\\
                 & \qquad\verb|const XMLBufferWriter& file_xml, const string& path)|\\
                 & \verb|void QDPRecordFileWriter::open(const XMLBufferWriter& file_xml, |\\
                 & \qquad\verb|const string& path)|\\
  \hline
  Close          & \verb|QDPRecordFileWriter::~QDPRecordFileWriter()|\\
                 & \verb|void QDPRecordFileWriter::close()|\\
  \hline
  Open?          & \verb|bool QDPRecordFileWriter::is_open()| \\
  \hline
  Errors?        & \verb|bool QDPRecordFileReader::bad()| \\
  \hline
  Write a \verb|T| & \verb|void write(QDPRecordFileWriter&, XMLBufferWriter& rec_xml, const T&)| \\
  Array of \verb|T| & \verb|void vwrite(QDPRecordFileWriter&, XMLBufferWriter& rec_xml, |\\
                 & \qquad\verb|const multi1d<T>&)| \\
  \hline
 \end{tabular}
\end{flushleft}
%
Base class for the various derived classes. Here, \verb|vwrites|
writes the sites as the slowest varying index and the array indices
(muli1d) inside of them. \verb|bad| states if any fatal errors have
occurred.

\paragraph{QDP record writer implementation classes}

\begin{flushleft}
  \begin{tabular}{|l|l|}
  \hline
  Single file written serially & \verb|QDPSingleFileSerialWriter| \\
  Single file written in parallel & \verb|QDPSingleFileParallelWriter| \\
  \hline
  Multi-dump file written serially & \verb|QDPMultiDumpSerialWriter| \\
  Multi-dump file written in parallel & \verb|QDPMultiDumpParallelWriter| \\
  \hline
 \end{tabular}
\end{flushleft}

{\bf Implementation note: QDP++ currently has only QDPSerialFileWriter in place
of QDPSingleFileSerialWriter}

%
Here is an example of how to use the record I/O facility.
%
\begin{small}
\begin{verbatim}
XMLBufferWriter file_xml;
QDPSingleFileSerialWriter out(file_xml, "foo");
XMLBufferWriter rec_xml;
LatticeColorMatrix a;
out.write(rec_xml, a);
write(out, rec_xml, a);  // for fun, write field twice
out.close();

QDPSingleFileSerialReader in(file_xml, "foo");
read(in, rec_xml, a);  // have choice of style
in.skip();
in.close();
\end{verbatim}
\end{small}


\newpage

%%%%%%%%%%%%%%%%%%%%%%%%%%%%%%%%%%%%%%%%%%%%%%%%%%%%%%%%%%%%%%%%%%%%%%
\section{Compilation with QDP}
%%%%%%%%%%%%%%%%%%%%%%%%%%%%%%%%%%%%%%%%%%%%%%%%%%%%%%%%%%%%%%%%%%%%%%

\subsection{Generic header and macros}

The compilation parameters:\\
\noindent
$Nd$ -- the number of space-time dimensions\\
\noindent
$Nc$ -- the dimension of the color vector space\\
\noindent
$Ns$ -- the dimension of the spin vector space\\
\noindent
are defined in {\tt qdp++/include/params.h} . 
There are macros {\tt ND}, {\tt NC},
{\tt NS} that are used to set the above parameters via
\begin{verbatim}
const int Nd = ND;
const int Nc = NC;
const int Ns = NS;
\end{verbatim}
%
\noindent
They are set in the build directories file {\tt include/qdp\_config.h} 
during configuration.

\subsection{Nonuniform color and precision}

Users wishing to vary color and precision within a single calculation
must use specific type names whenever these types
and names differ from the prevailing precision and color. 
Type declarations can be found in {\tt qdp++/include/defs.h} . A convenient
definition of a \tLatticeColorMatrix and \tLatticeDiracFermion is as follows:

\begin{verbatim}
typedef OLattice<PScalar<ColorMatrix<Complex<float>, Nc> > > LatticeColorMatrix
typedef OLattice<SpinVector<ColorVector<Complex<float>, Nc>, Ns> > LatticeFermion
\end{verbatim}

However, for the user to choose a specific number of colors:

\begin{verbatim}
const int NN = 17  // work in SU(17)
typedef OLattice<PScalar<ColorMatrix<Complex<float>, NN> > > LatticeColorMatrix17
\end{verbatim}


\newpage

%%%%%%%%%%%%%%%%%%%%%%%%%%%%%%%%%%%%%%%%%%%%%%%%%%%%%%%%%%%%%%%%%%%%%%
\section{Spin Conventions}
%%%%%%%%%%%%%%%%%%%%%%%%%%%%%%%%%%%%%%%%%%%%%%%%%%%%%%%%%%%%%%%%%%%%%%
\label{sec:spin}

The following set of $\gamma$-matrices are used in four dimensions:
\[
\begin{array}{lccc}
\gamma_0\quad\quad&
 \left(\begin{array}{rrrr}
   0&0&0&i\\
   0&0&i&0\\
   0&-i&0&0\\
   -i&0&0&0\\
 \end{array}\right)\quad &
 \left(\begin{array}{rr}
   0&i\sigma^1\\
   -i\sigma^1&0
 \end{array}\right)\quad\quad &
 -\sigma^2\!\otimes\!\sigma^1\\
\gamma_1\quad\quad&
 \left(\begin{array}{rrrr}
   0&0&0&-1\\
   0&0&1&0\\
   0&1&0&0\\
   -1&0&0&0\\
 \end{array}\right)\quad &
 \left(\begin{array}{rr}
   0&-i\sigma^2\\
   i\sigma^2&0
 \end{array}\right)\quad\quad &
 \sigma^2\!\otimes\!\sigma^2\\
\gamma_2\quad\quad&
 \left(\begin{array}{rrrr}
   0&0&i&0\\
   0&0&0&-i\\
   -i&0&0&0\\
   0&i&0&0\\
 \end{array}\right)\quad &
 \left(\begin{array}{rr}
   0&i\sigma^3\\
   -i\sigma^3&0
 \end{array}\right)\quad\quad &
 -\sigma^2\!\otimes\!\sigma^3\\
\gamma_3\quad\quad&
 \left(\begin{array}{rrrr}
   0&0&1&0\\
   0&0&0&1\\
   1&0&0&0\\
   0&1&0&0\\
 \end{array}\right)\quad &
 \left(\begin{array}{rr}
   0&\mathbf{1}\\
   \mathbf{1}&0
 \end{array}\right)\quad\quad &
 \sigma^1\!\otimes\!1\\
\end{array}
\]
%
The basis is chiral. All the possible gamma matrix products are
represented via
\[
\Gamma(n) = \gamma_0^{n_0} \gamma_1^{n_1} \gamma_2^{n_2} \gamma_3^{n_3}
\]
%
where $n_i$ are single bit fields.  Since $\gamma_0$ comes first the
bit for it must come first. %
So, $\gamma_5 = \gamma_0\gamma_1\gamma_2\gamma_3$ 
is represented as 1111b = 15d, and 
$\gamma_0\gamma_1\gamma_3$ is represented as 1011b = 11d
(note the ordering). The conventional $\gamma$-matrices are
\begin{eqnarray}
\Gamma(1) &=& \gamma_0\nonumber\\
\Gamma(2) &=& \gamma_1\nonumber\\
\Gamma(4) &=& \gamma_2\nonumber\\
\Gamma(8) &=& \gamma_3\nonumber
\end{eqnarray}

This enumeration is $\gamma$-basis independent.


\newpage

%%%%%%%%%%%%%%%%%%%%%%%%%%%%%%%%%%%%%%%%%%%%%%%%%%%%%%%%%%%%%%%%%%%%%%
\section{Implementation Details}
%%%%%%%%%%%%%%%%%%%%%%%%%%%%%%%%%%%%%%%%%%%%%%%%%%%%%%%%%%%%%%%%%%%%%%
\label{sec:implementation}

The following table lists some of the QDP headers.
\begin{center}
\begin{tabular}{|l|l|}
\hline
 name & purpose \\
\hline
 {\tt qdp.h        } & Master header and QDP utilities     \\
 {\tt qdptype.h    } & Main class definition               \\
 {\tt qdpexpr.h    } & Expression class definition         \\
 {\tt primitive.h  } & Main header for all primitive types \\
 {\tt primscalar.h } & Scalar primitive class and operations \\
 {\tt primmatrix.h } & Matrix primitive and operations \\
 {\tt primvector.h } & Vector primitive and operations \\
 {\tt primseed.h   } & Seed (random number) primitive  \\
 {\tt reality.h    } & Complex number internal class  \\
 {\tt simpleword.h } & Machine word-type operations  \\
\hline
\end{tabular}
\end{center}

\newpage

%%%%%%%%%%%%%%%%%%%%%%%%%%%%%%%%%%%%%%%%%%%%%%%%%%%%%%%%%%%%%%%%%%%%%%
\section{Supported Operations}
%%%%%%%%%%%%%%%%%%%%%%%%%%%%%%%%%%%%%%%%%%%%%%%%%%%%%%%%%%%%%%%%%%%%%%
\label{sec:supported}

This section describes in some detail the names and functionality for
all functions in the interface involving linear algebra with and
without shifts.  

All QDP objects are of type QDPType, and QDP functions act on objects
of this base class type. Unless otherwise indicated, operations occur
on all sites in the specified subset of the target, often an
assignment statement or object definition. The indexing of a QDPType
returns an lvalue suitable for assignment (but not object
definition). It is also used to narrow the lattice sites participating
in a global reduction since the result of such a reduction is a
lattice scalar, hence are independent of lattice sites.

Supported operations are listed below.
Convention: protoyypes are basically of the form:

\begin{verbatim}
QDPType  unary_function(const QDPType&)
QDPType  binary_function(const QDPType&, const QDPType&)
\end{verbatim}

\medskip

\subsection{Subsets and Maps}
\label{sec:subsets}

\begin{flushleft}
\begin{tabular}{lp{4.0in}}
{\tt Set::make(const SetFunc\&)} & Set construction of ordinality num subsets.
                               func maps coordinates to a coloring in [0,num)\\

{\tt Map::make(const MapFunc\&)} & Construct a map function from source sites to the dest site.\\
\end{tabular}
\end{flushleft}

\medskip

\subsection{Infix operators}
\label{sec:infix}

\noindent
{\em Unary infix (e.g., {\tt operator-}):}

\begin{flushleft}
\begin{tabular}{lcp{4.0in}}
\verb|-| &: & negation\\
\verb|+| &: & unary plus\\
\verb|~| &: & bitwise not\\
\verb|!| &: & boolean not\\
\end{tabular}
\end{flushleft}

\noindent
{\em Binary infix (e.g., {\tt operator+}):}

\begin{flushleft}
\begin{tabular}{lcp{4.0in}}
\verb|+|  &:& addition\\
\verb|-|  &:& subtraction\\
\verb|*|  &:& multiplication\\
\verb|/|  &:& division\\
\verb|%|  &:& mod\\
\verb|&|  &:& bitwise and\\
{\tt |}  &:& bitwise or\\
\verb|^|  &:& bitwise exclusive or\\
\verb|<<| &:& left-shift\\
\verb|>>| &:& right-shift\\
\end{tabular}
\end{flushleft}

\noindent
{\em Comparisons (returning booleans, e.g., {\tt operator<}):}

\begin{verbatim}
<, <=, >, >=, ==, !=
&& : and of 2 booleans
|| : or of 2 boolean
\end{verbatim}


\noindent
{\em Assignments (e.g., {\tt operator+=}):}

\begin{verbatim}
=, +=, -=, *=, /=, %=, |=, &=, ^=, <<=, >>=
\end{verbatim}

\noindent
{\em Trinary:}

\begin{verbatim}
where(bool,arg1,arg2) : the C trinary "?" operator -> (bool) ? arg1 : arg2
\end{verbatim}

\medskip


\subsection{Functions (standard C math lib)}
\label{sec:cfuncs}

\noindent
{\em Unary:}

\begin{verbatim}
cos, sin, tan, acos, asin, atan, cosh, sinh, tanh,
exp, log, log10, sqrt,
ceil, floor, fabs
\end{verbatim}

\noindent
{\em Binary:}

\begin{verbatim}
ldexp, pow, fmod, atan2
\end{verbatim}

\medskip


\subsection{Additional functions (specific to QDP)}
\label{sec:funcs}

\noindent
{\em Unary:}

\begin{flushleft}
\begin{tabular}{lcp{4.0in}}
\verb|adj|              &:& hermitian conjugate (adjoint)\\
\verb|conj|             &:& complex conjugate\\
\verb|transpose|        &:& matrix tranpose, on a scalar it is a nop\\
\verb|trace|            &:& matrix trace\\
\verb|real|             &:& real part\\
\verb|imag|             &:& imaginary part\\
\verb|colorTrace|       &:& trace over color indices\\
\verb|spinTrace|        &:& trace over spin indices\\
\verb|timesI|           &:& multiplies argument by imag "i"\\
\verb|localNorm2|       &:& on fibers computes trace(adj(source)*source)\\
\end{tabular}
\end{flushleft}

\noindent
{\em Binary}:

\begin{flushleft}
\begin{tabular}{lcp{4.0in}}
\verb|cmplx|              &:& returns complex object   arg1 + i*arg2\\
\verb|localInnerProduct|  &:& at each site computes trace(adj(arg1)*arg2)\\
\verb|outerProduct|       &:& at each site constructs $({\rm arg1}_i * {\rm arg2}_j^*)_{ij}$\\
\end{tabular}
\end{flushleft}

\subsection{In place functions}
\label{sec:inplace}

\begin{flushleft}
\begin{tabular}{lcp{4.0in}}
\verb|random(dest)|            &:& uniform random numbers - all components\\
\verb|gaussian(dest)|          &:& uniform random numbers - all components\\
\verb|copymask(dest,mask,src)| &:& copy src to dest under boolean mask\\
\end{tabular}
\end{flushleft}

%zero(dest)              : zero out all elements

\medskip

\subsection{Broadcasts}
\label{sec:broadcasts}

{\em Broadcasts via assignments (via, {\tt operator=}):}
\begin{flushleft}
\begin{tabular}{lcp{4.0in}}
\verb|<LHS> = <constant>| &:& globally set conforming LHS to constant\\
\verb|<LHS> = zero|       &:& global always set LHS to zero\\
\end{tabular}
\end{flushleft}

\medskip

\subsection{Global reductions}
\label{sec:reductions}

\begin{flushleft}
\begin{tabular}{lcp{4.0in}}
\verb|sum(arg1)|              &:& sum over lattice indices returning 
                                  object of same fiber type\\
\verb|norm2(arg1)|            &:& sum(localNorm2(arg1))\\
\verb|innerProduct(arg1,arg2)|&:& sum(localInnerProduct(arg1,arg2))\\
\verb|sumMulti(arg1,Set)|     &:& sum over each subset of Set returning \#subset
                                  objects of same fiber type\\
\end{tabular}
\end{flushleft}

\medskip

\subsection{Accessors}
\label{sec:peek}

Peeking and poking (accessors) into various component indices of objects.

\begin{flushleft}
\begin{tabular}{lcp{4.0in}}
\verb|peekSite(arg1,multi1d<int> coords)|&:& return object located at lattice coords\\
\verb|peekColor(arg1,int row,int col)|   &:& return color matrix elem row and col\\
\verb|peekColor(arg1,int row)|           &:& return color vector elem row\\
\verb|peekSpin(arg1,int row,int col)|   &:& return spin matrix elem row and col\\
\verb|peekSpin(arg1,int row)|            &:& return spin vector elem row\\
&&\\
\verb|pokeSite(dest,src,multi1d<int> coords)|&:& insert into site given by coords\\
\verb|pokeColor(dest,src,int row,int col)|   &:& insert into color matrix elem row and col\\
\verb|pokeColor(dest,src,int row)|           &:& insert into color vector elem row\\
\verb|pokeSpin(dest,src,int row,int col)|   &:& insert into spin matrix elem row and col\\
\verb|pokeSpin(dest,src,int row)|            &:& insert into spin vector elem row\\
\end{tabular}
\end{flushleft}


\medskip

\subsection{More exotic functions:}
\label{sec:otherfuncs}

\noindent
Applies spin projection  $(1 + isign*\gamma_\mu)$*\verb|psi|
returning a half spin vector or matrix
\begin{flushleft}
\verb|spinProject(QDPType psi, int dir, int isign)|
\end{flushleft}

\noindent
Applies spin reconstruction of $(1 + isign*\gamma_\mu)$*\verb|psi|
returning a full spin vector or matrix
\begin{flushleft}
\verb|spinReconstruct(QDPType psi, int dir, int isign)|
\end{flushleft}

\bigskip

\subsection{Operations on subtypes}

Types in the QDP interface are parameterized by a variety of types, and
can look like the following:

\begin{small}
\begin{verbatim}
typedef OLattice<PScalar<PColorMatrix<RComplex<float>, Nc> > > LatticeColorMatrix
typedef OLattice<PSpinVector<PColorVector<RComplex<float>, Nc>, Ns> > LatticeFermion
\end{verbatim}
\end{small}

\begin{itemize}
\item {\em Word type}: 
  int, float, double, bool. Basic machine types.
\item {\em Reality type}: 
  RComplex or RScalar. 
\item {\em Primitive type}: 
  PScalar, PVector, PMatrix, PSeed. 
\item {\em Inner grid type}: 
  IScalar or ILattice. 
\item {\em Outer grid type}:
  OScalar or OLattice. 
\end{itemize}

\noindent
Supported operations for each type level as follows:

\paragraph{\bf\em Grid type:} {\em OScalar, OLattice, IScalar, ILattice}\\
All operations listed in Sections~\ref{sec:infix}--\ref{sec:otherfuncs}

\paragraph{\bf\em Primitive type:}
%\begin{itemize}
%\item
\paragraph{\bf\em PScalar:}
All operations listed in Sections~\ref{sec:infix}--\ref{sec:otherfuncs}

%\item
\paragraph{\bf\em PMatrix$<$N$>$:}
\begin{flushleft}
  \begin{tabular}{lp{5.0in}}
  {\em Unary}:& {\tt -(PMatrix)}, {\tt +(PMatrix)}\\
  {\em Binary}:& {\tt -(PMatrix,PMatrix)}, {\tt +(PMatrix,PMatrix)}, 
       {\tt *(PMatrix,PScalar)}, {\tt *(PScalar,PMatrix)}, {\tt *(PMatrix,PMatrix)}\\
  {\em Comparisons}:& none\\
  {\em Assignments}:& {\tt =(PMatrix)}, {\tt =(PScalar)}, {\tt -=(PMatrix)}, 
       {\tt +=(PMatrix)}, {\tt *=(PScalar)}\\
  {\em Trinary}:& {\tt where}\\
  {\em C-lib funcs}:& none\\
  {\em QDP funcs}:& all\\
  {\em In place funcs}:& all\\
  {\em Reductions}:& all\\
  \end{tabular}
\end{flushleft}

%\item
\paragraph{\bf\em PVector$<$N$>$:}
\begin{flushleft}
  \begin{tabular}{lp{5.0in}}
  {\em Unary}:& {\tt -(PVector)}, {\tt +(PVector)}\\
  {\em Binary}:& {\tt -(PVector,PVector)}, {\tt +(PVector,PVector)}, 
       {\tt *(PVector,PScalar)}, {\tt *(PScalar,PVector)}, {\tt *(PMatrix,PVector)}\\
  {\em Comparisons}:& none\\
  {\em Assignments}:& {\tt =(PVector)}, {\tt -=(PVector)}, {\tt +=(PVector)}, 
       {\tt *=(PScalar)}\\
  {\em Trinary}:& {\tt where}\\
  {\em C-lib funcs}:& none\\
  {\em QDP funcs}:& {\tt real}, {\tt imag}, {\tt timesI}, 
        {\tt localNorm2}, {\tt cmplx}, {\tt localInnerProduct}, {\tt outerProduct}\\
  {\em In place funcs}:& all\\
  {\em Broadcasts}:& {\tt =(Zero)}\\
  {\em Reductions}:& all\\
  \end{tabular}
\end{flushleft}

%\item
\paragraph{\bf\em PSpinMatrix$<$N$>$:}
  Inherits same operations as PMatrix
\begin{flushleft}
  \begin{tabular}{lp{5.0in}}
  {\em Unary}:& {\tt spinTrace}\\
  {\em Binary}:& {\tt *(PSpinMatrix,Gamma)}, {\tt *(Gamma,PSpinMatrix)}\\
  {\em Exotic}:& {\tt peekSpin}, {\tt pokeSpin}, {\tt spinProjection}, 
      {\tt spinReconstruction}\\
  \end{tabular}
\end{flushleft}

%\item
\paragraph{\bf\em PSpinVector$<$N$>$:}
  Inherits same operations as PVector
\begin{flushleft}
  \begin{tabular}{lp{5.0in}}
  {\em Binary}:& {\tt *(Gamma,PSpinVector)}\\
  {\em Exotic}:& {\tt peekSpin}, {\tt pokeSpin}, {\tt spinProjection}, 
      {\tt spinReconstruction}\\
  \end{tabular}
\end{flushleft}

%\item
\paragraph{\bf\em PColorMatrix$<$N$>$:}
  Inherits same operations as PMatrix
\begin{flushleft}
  \begin{tabular}{lp{5.0in}}
  {\em Unary}:& {\tt colorTrace}\\
  {\em Binary}:& {\tt *(PColorMatrix,Gamma)}, {\tt *(Gamma,PColorMatrix)}\\
  {\em Exotic}:& {\tt peekColor}, {\tt pokeColor}\\
  \end{tabular}
\end{flushleft}

%\item
\paragraph{\bf\em PColorVector$<$N$>$:}
  Inherits same operations as PVector
\begin{flushleft}
  \begin{tabular}{lp{5.0in}}
  {\em Binary}:& {\tt *(Gamma,PColorVector)}\\
  {\em Exotic}:& {\tt peekColor}, {\tt pokeColor}\\
  \end{tabular}
\end{flushleft}

%\end{itemize}  % end primitive

\paragraph{\bf\em Reality:} {\em RScalar, RComplex}\\
All operations listed in Sections~\ref{sec:infix}--\ref{sec:otherfuncs}

\paragraph{\bf\em Word:} {\em int, float, double, bool}\\
All operations listed in Sections~\ref{sec:infix}--\ref{sec:otherfuncs}.
Only boolean ops allowed on bool.

\newpage

%%%%%%%%%%%%%%%%%%%%%%%%%%%%%%%%%%%%%%%%%%%%%%%%%%%%%%%%%%%%%%%%%%%%%%
\section{Detailed function description}
%%%%%%%%%%%%%%%%%%%%%%%%%%%%%%%%%%%%%%%%%%%%%%%%%%%%%%%%%%%%%%%%%%%%%%
\label{sec:details}

The purpose of this section is to show some explicit prototypes and
usages for the functions described in Section~\ref{sec:supported}.
In that section, all the functions are shown with complete information
on which operations and their meaning are supported on some combination
of types. The purpose of this section is something like the inverse -
namely show all the functions and what are some (selected) usages.

\subsection{Unary Operations}

\paragraph{Elementary unary functions on reals}

\begin{flushleft}
  \begin{tabular}{|l|l|}
  \hline
  Syntax      & \protoUnary  \\
  \hline
  Meaning     & $r = {\rm func}(a)$ \\
  \hline
  {\it func}  & cos, sin, tan, acos, asin, atan,
		sqrt, abs, exp, log, sign \\
  \hline
  \itt        & \tReal, \tLatticeReal \\
  \hline
  \end{tabular}
\end{flushleft}

%\end{document}

\paragraph{Elementary unary functions on complex values}

\begin{flushleft}
  \begin{tabular}{|l|l|}
  \hline
  Syntax      & \protoUnary  \\
  \hline
  Meaning     & $r = {\rm func}(a)$ \\
  \hline
  {\it func}  & exp, sqrt, log \\
  \hline
  \itt        & \tComplex, \tLatticeComplex \\
  \hline
  \end{tabular}
\end{flushleft}

\paragraph{Assignment operations}

\begin{flushleft}
  \begin{tabular}{|l|l|}
  \hline
  Syntax      & \protoAssignmentQual{operator=} \\
  \hline
  Meaning     & $r = a$ \\
  \hline
  \itt     & \allNumericTypes \\
  \hline
  \end{tabular}
\end{flushleft}

\paragraph{Shifting}

\begin{flushleft}
  \begin{tabular}{|l|l|}
  \hline
  Syntax      & \protoUnaryShift  \\
  \hline
  Meaning     & $r = a$ \\
  \hline
  \itt     & \allNumericTypes \\
  \hline
  \end{tabular}
\end{flushleft}


\paragraph{Hermitian conjugate}

\begin{flushleft}
  \begin{tabular}{|l|l|}
  \hline
  Syntax      & \protoUnaryQual{\tt adj}  \\
  \hline
  Meaning     & $r = a^\dagger$ \\
  \hline
  \itt        & \tReal, \tComplex, \tColorMatrix, \tDiracPropagator \\
              & \simLatticeVariants \\
  \hline
  \end{tabular}
\end{flushleft}

\paragraph{Transpose}

\begin{flushleft}
  \begin{tabular}{|l|l|}
  \hline
  Syntax      & \protoUnaryQual{\tt transpose}  \\
  \hline
  Meaning     & $r = {\rm transpose}(a)$ \\
  \hline
  \itt        & \tReal, \tComplex, \tColorMatrix, \tDiracPropagator \\
              & \simLatticeVariants \\
  \hline
  \end{tabular}
\end{flushleft}

\paragraph{Complex conjugate}

\begin{flushleft}
  \begin{tabular}{|l|l|}
  \hline
  Syntax      & \protoUnaryQual{\tt conj}  \\
  \hline
  Meaning     & $r = a^*$ \\
  \hline
  \itt        & \tReal, \tComplex, \tColorMatrix, \tDiracFermion, \tDiracPropagator \\
              & \simLatticeVariants \\
  \hline
  \end{tabular}
\end{flushleft}

\subsection{Type conversion}

Types can be precision converted via a conversion function of the destination
class.

\paragraph{Convert integer or float to double}
\begin{flushleft}
  \begin{tabular}{|l|l|}
  \hline
  Syntax      & {\it Type2 Type2}({\tt const {\it Type1}\& a}) \\
  \hline
  Example     & {\tt \tLatticeReal{} a; \tLatticeRealD{} r = \tLatticeRealD(a)} \\
              & {\tt \tLatticeColorMatrix{} a; \tLatticeColorMatrixD{} r = \tLatticeColorMatrixD(a)} \\
  \hline
  {\it Type1} & All single precision numeric types \\
  {\it Type2} & All conforming double precision numeric types \\
  \hline
  \end{tabular}
\end{flushleft}

\paragraph{Convert double to float}
\begin{flushleft}
  \begin{tabular}{|l|l|}
  \hline
  Syntax      & {\it Type2 Type2}({\tt const {\it Type1}\& a})\\
  \hline
  Example     & {\tt \tLatticeRealD{} a; \tLatticeReal{} r = \tLatticeReal(a)} \\
              & {\tt \tLatticeColorMatrixD{} a; \tLatticeColorMatrix{} r = \tLatticeColorMatrix(a)} \\
  \hline
  {\it Type1} & All double precision numeric types \\
  {\it Type2} & All conforming single precision numeric types \\
  \hline
  \end{tabular}
\end{flushleft}

\paragraph{Integer to real}

\begin{flushleft}
  \begin{tabular}{|l|l|}
  \hline
  Syntax      & {\it Type2 Type2}({\tt const {\it Type1}\& a})\\
  \hline
  Example     & {\tt \tLatticeInt{} a; \tLatticeReal{} r = \tLatticeReal(a)} \\
  \hline
  {\it Type1} & All integer precision numeric types \\
  {\it Type2} & All conforming real precision numeric types \\
  \hline
  \end{tabular}
\end{flushleft}

\paragraph{Real to integer}

\begin{flushleft}
  \begin{tabular}{|l|l|}
  \hline
  Syntax      & {\it Type2 Type2}({\tt const {\it Type1}\& a})\\
  \hline
  Example     & {\tt \tLatticeReal{} a; \tLatticeInt{} r = \tLatticeInt(a)} \\
  \hline
  \end{tabular}
\end{flushleft}

\paragraph{Real to float} 

\begin{flushleft}
  \begin{tabular}{|l|l|}
  \hline
  Syntax      & {\tt float toFloat(const \tReal\& a)} \\
  \hline
  Meaning     & \verb|r = float(a);|\\
  \hline
  Example     & {\tt \tReal{} a; float r = toFloat(a);} \\
  \hline
  \end{tabular}
\end{flushleft}

The QDP type \tReal{} is not a primitive type, so an explicit conversion is provided.

\paragraph{Double to double} 

\begin{flushleft}
  \begin{tabular}{|l|l|}
  \hline
  Syntax      & {\tt double toDouble(const \tRealD\& a)} \\
  \hline
  Meaning     & \verb|r = double(a);|\\
  \hline
  Example     & {\tt \tRealD{} a; double r = toDouble(a);} \\
  \hline
  \end{tabular}
\end{flushleft}

The QDP type \tRealD{} is not a primitive type, so an explicit conversion is provided.

\paragraph{Bool to bool} 

\begin{flushleft}
  \begin{tabular}{|l|l|}
  \hline
  Syntax      & {\tt bool toBool(const \tBoolean\& a)} \\
  \hline
  Meaning     & \verb|r = bool(a);|\\
  \hline
  Example     & {\tt \tBoolean{} a; bool r = toBool(a);} \\
  \hline
  \end{tabular}
\end{flushleft}

The QDP type \tBoolean{} is not a primitive type, so an explicit conversion is provided.

\subsection{Operations on complex arguments}

\paragraph{Convert real and imaginary to complex}

\begin{flushleft}
  \begin{tabular}{|l|l|}
  \hline
  Syntax      & \protoBinaryQual{cmplx} \\
  \hline
  Meaning     & $\Re r = a$, $\Im r = b$ \\
  \hline
  {\it Type1} & \tConstant, \tReal, \simLatticeVariants\\
  {\it Type2} & \tConstant, \tReal, \simLatticeVariants\\
  \hline
  {\it Type}  & \tComplex, \simLatticeVariants\\
  \hline
  Example     & {\tt \tReal a;}\\
              & {\tt \tComplex{} = cmplx(a, 0);}\\
  \hline
  \end{tabular}
\end{flushleft}

\paragraph{Real part of complex}

\begin{flushleft}
  \begin{tabular}{|l|l|}
  \hline
  Syntax      & \protoUnaryQual{\tt real}  \\
  \hline
  Meaning     & $r = \Re a$ \\
  \hline
  \end{tabular}
\end{flushleft}

\paragraph{Imaginary part of complex}

\begin{flushleft}
  \begin{tabular}{|l|l|}
  \hline
  Syntax      & \protoUnaryQual{\tt imag}  \\
  \hline
  Meaning     & $r = \Im a$ \\
  \hline
  \end{tabular}
\end{flushleft}

\subsection{Component extraction and insertion}
\paragraph{Accessing a site object}

\begin{flushleft}
  \begin{tabular}{|l|l|}
  \hline
  Syntax      & {\it Type} \verb|peekSite|({\tt const Lattice{\it Type}\& a, const multi1d<int>\& c})\\
  \hline
  Meaning     & $r = a[x]$\\
  \hline
  \end{tabular}
\end{flushleft}

\paragraph{Accessing a color matrix element}

\begin{flushleft}
  \begin{tabular}{|l|l|}
  \hline
  Syntax      & {\tt \tLatticeComplex} \verb|peekColor|({\tt const \tLatticeColorMatrix\& a},\\
              &\quad         {\tt int i, int j})\\
              & {\tt \tLatticeSpinMatrix} \verb|peekColor|({\tt const \tLatticeDiracPropagator\& a},\\
              &\quad         {\tt int i, int j})\\
  \hline
  Meaning     & $r = a_{i,j}$\\
  \hline
  \end{tabular}
\end{flushleft}

\paragraph{Inserting a color matrix element}

\begin{flushleft}
  \begin{tabular}{|l|l|}
  \hline
  Syntax      & {\tt LatticeColorMatrix\&} \verb|pokeColor|({\tt \tLatticeColorMatrix\& r,}\\
              &   \quad{\tt const \tLatticeComplex\& a, int i, int j})\\
  \hline
  Meaning     & $r_{i,j} = a$\\
  \hline
  \end{tabular}
\end{flushleft}

\paragraph{Accessing a color vector element}

\begin{flushleft}
  \begin{tabular}{|l|l|}
  \hline
  Syntax      & {\tt \tLatticeComplex} \verb|peekColor|({\tt const \tLatticeColorVector\& a},\\
              &\quad         {\tt int i})\\
              & {\tt \tLatticeSpinVector} \verb|peekColor|({\tt const \tLatticeDiracFermion\& a},\\
              &\quad         {\tt int i})\\
  \hline
  Meaning     & $r = a_{i}$\\
  \hline
  \end{tabular}
\end{flushleft}

This function will extract the desired color component with all the
other indices unchanged.

A lattice color vector is another name (typedef) for a
\tLatticeStaggeredFermion. Namely, an object that is vector in color
spin and a scalar in spin space.  Together with spin accessors, one
can build a \tLatticeDiracFermion.

\paragraph{Inserting a color vector element}

\begin{flushleft}
  \begin{tabular}{|l|l|}
  \hline
  Syntax      & {\tt \tLatticeColorVector\&} \verb|pokeColor|({\tt \tLatticeColorVector\& r},\\
              &   \qquad{\tt const \tLatticeComplex\& a, int i})\\
  \hline
  Meaning     & $r_{i} = a$\\
  \hline
  \end{tabular}
\end{flushleft}

This function will extract the desired color component with all the
other indices unchanged.

A lattice color vector is another name (typedef) for a
\tLatticeStaggeredFermion. Namely, an object that is vector in color
spin and a scalar in spin space.  Together with spin accessors, one
can build a \tLatticeDiracFermion{} or a \tLatticeDiracPropagator.

\paragraph{Accessing a spin matrix element}

\begin{flushleft}
  \begin{tabular}{|l|l|}
  \hline
  Syntax      & {\tt \tLatticeComplex} \verb|peekSpin|({\tt const \tLatticeSpinMatrix\& a},\\
              &\quad         {\tt int i, int j})\\
              & {\tt \tLatticeColorMatrix} \verb|peekSpin|({\tt const \tLatticeDiracPropagator\& a},\\
              &\quad         {\tt int i, int j})\\
  \hline
  Meaning     & $r = a_{i,j}$\\
  \hline
  \end{tabular}
\end{flushleft}

\paragraph{Inserting a spin matrix element}

\begin{flushleft}
  \begin{tabular}{|l|l|}
  \hline
  Syntax      & {\tt \tLatticeSpinMatrix\&} \verb|pokeSpin|({\tt \tLatticeSpinMatrix\& r},\\
              &   \quad{\tt const \tLatticeComplex\& a, int i, int j})\\
  \hline
  Meaning     & $r_{i,j} = a$\\
  \hline
  \end{tabular}
\end{flushleft}

\paragraph{Accessing a spin vector element}

\begin{flushleft}
  \begin{tabular}{|l|l|}
  \hline
  Syntax      & {\tt \tLatticeComplex} \verb|peekSpin|({\tt const \tLatticeSpinVector\& a},\\
              &\quad         {\tt int i})\\
              & {\tt \tLatticeColorVector} \verb|peekSpin|({\tt const \tLatticeDiracFermion\& a},\\
              &\quad         {\tt int i})\\
  \hline
  Meaning     & $r = a_{i}$\\
  \hline
  \end{tabular}
\end{flushleft}

This function will extract the desired spin component with all the
other indices unchanged.

A lattice spin vector is an object that is a vector in spin space and
a scalar in color space.  Together with color accessors, one can build
a \tLatticeDiracFermion{} or a \tLatticeDiracPropagator.

\paragraph{Inserting a spin vector element}

\begin{flushleft}
  \begin{tabular}{|l|l|}
  \hline
  Syntax      & {\tt \tLatticeSpinVector\&} \verb|pokeSpin|({\tt \tLatticeSpinVector\& r},\\
              &   \quad{\tt const \tLatticeComplex\& a, int i})\\
  \hline
  Meaning     & $r_{i} = a$\\
  \hline
  \end{tabular}
\end{flushleft}

This function will extract the desired spin component with all the
other indices unchanged.

A lattice spin vector is an object that is a vector in spin space and
a scalar in color space.  Together with color accessors, one can build
a \tLatticeDiracFermion{} or a \tLatticeDiracPropagator.

\paragraph{Trace of matrix}

\begin{flushleft}
  \begin{tabular}{|l|l|}
  \hline
  Syntax      & {\it Type2} \verb|trace|({\tt const {\it Type1}\& a})\\
  \hline
  Meaning     & $r = \Tr a$ \\
  \hline
  {\it Type1} & \tColorMatrix, \tDiracPropagator, \simLatticeVariants\\
  {\it Type2} & \tComplex, \tComplex, \simLatticeVariants\\
  \hline
  Example     & {\tt \tLatticeColorMatrix{} a;}\\
              & {\tt \tLatticeComplex{} r = trace(a);} \\
  \hline
  \end{tabular}
\end{flushleft}

Traces over all matrix indices. It is an error to trace over a vector index.
It will trivially trace a scalar variable.

\paragraph{Color trace of matrix}

\begin{flushleft}
  \begin{tabular}{|l|l|}
  \hline
  Syntax      & {\it Type2} \verb|traceColor|({\tt const {\it Type1}\& a})\\
  \hline
  Meaning     & $r = \Tr a$ \\
  \hline
  {\it Type1} & \tSpinMatrix, \simLatticeVariants\\
  {\it Type2} & \tComplex, \simLatticeVariants\\
  \hline
  \hline
  Example     & {\tt \tLatticeDiracPropagator{} a;}\\
              & {\tt \tLatticeSpinMatrix{} r = traceColor(a);} \\
  \hline
  \end{tabular}
\end{flushleft}

Traces only over color matrix indices. It is an error to trace over a color vector index.
All other indices are left untouched.
It will trivially trace a scalar variable.

\paragraph{Spin trace of matrix}

\begin{flushleft}
  \begin{tabular}{|l|l|}
  \hline
  Syntax      & {\it Type2} \verb|traceSpin|({\tt const {\it Type1}\& a})\\
  \hline
  Meaning     & $r = \Tr a$ \\
  \hline
  {\it Type1} & \tDiracPropagator, \simLatticeVariants\\
  {\it Type2} & \tColorMatrix, \simLatticeVariants\\
  \hline
  \hline
  Example     & {\tt \tLatticeDiracPropagator{} a;}\\
              & {\tt \tLatticeColorMatrix{} r = traceSpin(a);} \\
  \hline
  \end{tabular}
\end{flushleft}

Traces only over spin matrix indices. It is an error to trace over a
spin vector index.  All other indices are left untouched.  It will
trivially trace a scalar variable.

\paragraph{Dirac spin projection}

\begin{flushleft}
  \begin{tabular}{|l|l|}
  \hline 
  Syntax      & {\it Type2} \verb|spinProject|({\tt const {\it Type1}\& a, int d, int p})\\
  \hline
  Meaning     & $r = (1 + p\gamma_d)a$\\
  \hline
  {\it Type1} & \tDiracFermion, \simLatticeVariants\\
  {\it Type2} & \tHalfFermion, \simLatticeVariants\\
  \hline
  \end{tabular}
\end{flushleft}

\paragraph{Dirac spin reconstruction}

\begin{flushleft}
  \begin{tabular}{|l|l|}
  \hline
  Syntax      & {\it Type2} \verb|spinReconstruct|({\tt const {\it Type1}\& a, int d, int p})\\
  \hline
  Meaning     & $r = {\rm recon\,}(p,d,a)$\\
  \hline
  {\it Type1} & \tHalfFermion, \simLatticeVariants\\
  {\it Type2} & \tDiracFermion, \simLatticeVariants\\
  \hline
  \end{tabular}
\end{flushleft}


\subsection{Binary Operations with Constants}

\paragraph{Multiplication by real constant}

\begin{flushleft}
  \begin{tabular}{|l|l|}
  \hline
  Syntax      & {\it Type} \verb|operator*|({\tt const \tReal\& a, const {\it Type}\& b})\\
              & {\it Type} \verb|operator*|({\tt const {\it Type}\& b, const \tReal\& a})\\
  \hline
  Meaning     & $r = a*b$ ($a$ real, constant)\\
  \hline
  \itt     & \allFloatTypes \\
  \hline
  \end{tabular}
\end{flushleft}

\paragraph{Multiplication by complex constant}

\begin{flushleft}
  \begin{tabular}{|l|l|}
  \hline
  Syntax      & {\it Type} \verb|operator*|({\tt const \tReal\& a, const {\it Type}\& b})\\
              & {\it Type} \verb|operator*|({\tt const {\it Type}\& b, const \tReal\& a})\\
  \hline
  Meaning     & $r = a*b$ ($a$ complex, constant)\\
  \hline
  \itt     & \allNumericTypes \\
  \hline
  \end{tabular}
\end{flushleft}

\paragraph{Left multiplication by gamma matrix}

\begin{flushleft}
  \begin{tabular}{|l|l|}
  \hline
  Syntax      & {\it Type} \verb|operator*|({\tt const {\it Gamma}\& a, const {\it Type}\& b})\\
  \hline
  Meaning     & $r = \gamma_d * a$ \\
  \hline
  {\it Gamma} & Gamma constructed from an explicit integer in $[0,N_s^2-1]$\\
  \itt        & \tSpinVector, \tSpinMatrix, \tHalfFermion, \tDiracFermion, \tDiracPropagator, \\
              & and similar lattice variants \\
  \hline
  Example     & \verb|r = Gamma(7) * b;|\\
  \hline
  \end{tabular}
\end{flushleft}

See Section~\ref{sec:spin} for details on $\gamma$-matrix conventions.


\paragraph{Right multiplication by gamma matrix}
\begin{flushleft}
  \begin{tabular}{|l|l|}
  \hline
  Syntax      & {\it Type} \verb|operator*|({\tt const {\it Type}\& a, const {\it Gamma}\& b})\\
  \hline
  Meaning     & $r = a * \gamma_d$ \\
  \hline
  {\it Gamma} & Gamma constructed from an explicit integer in $[0,N_s^2-1]$\\
  \itt        & \tSpinMatrix, \tDiracPropagator, and similar lattice variants \\
  \hline
  Example     & \verb|r = a * Gamma(15);|\\
  \hline
  \end{tabular}
\end{flushleft}

See Section~\ref{sec:spin} for details on $\gamma$-matrix conventions.

\subsection{Binary Operations with Fields}

\paragraph{Division of real fields}

\begin{flushleft}
  \begin{tabular}{|l|l|}
  \hline
  Syntax      & \protoBBinaryQual{operator/} \\
  \hline
  Meaning     & $r = a/b$ \\
  \hline
  \end{tabular}
\end{flushleft}

\paragraph{Addition}

\begin{flushleft}
  \begin{tabular}{|l|l|}
  \hline
  Syntax      & \protoBBinaryQual{operator+} \\
  \hline
  Meaning     & $r = a + b$ \\
  \hline
  \itt        & \allNumericTypes \\
  \hline
  \end{tabular}
\end{flushleft}

\paragraph{Subtraction}

\begin{flushleft}
  \begin{tabular}{|l|l|}
  \hline
  Syntax      & \protoBBinaryQual{operator-} \\
  \hline
  Meaning     & $r = a - b$ \\
  \hline
  \itt        & \allNumericTypes \\
  \hline
  \end{tabular}
\end{flushleft}

\paragraph{Multiplication: uniform types}

\begin{flushleft}
  \begin{tabular}{|l|l|}
  \hline
  Syntax      & \protoBBinaryQual{operator*} \\
  \hline
  Meaning     & $r = a * b$ \\
  \hline
  \itt        & \tConstant, \tReal, \tComplex, \tInt, \tColorMatrix, \tSpinMatrix, \tDiracPropagator\\
  \hline
  \end{tabular}
\end{flushleft}

\paragraph{ColorMatrix matrix from outer product}

\begin{flushleft}
  \begin{tabular}{|l|l|}
  \hline
  Syntax      & \protoBinaryQual{outerProduct} \\
  \hline
  Meaning     & $r_{i,j} = a_i * b_j^*$ \\
  \hline
  {\it Type1,2} & \tColorVector, \tLatticeColorVector\\
  \itt        & \tColorMatrix, \tLatticeColorMatrix\\
  \hline
  \end{tabular}
\end{flushleft}

\paragraph{Left multiplication by gauge matrix}

\begin{flushleft}
  \begin{tabular}{|l|l|}
  \hline
  Syntax      & {\it Type} \verb|operator*|({\tt const {\it Type1}\& a, const {\it Type}\& b})\\
  \hline
  Meaning     & $r = a * b$ \\
  \hline
  {\it Type1} & \tColorMatrix, \tLatticeColorMatrix\\
  \itt        & \tConstant, \tComplex, \tColorMatrix, \tColorVector, \\
              & \tSpinVector, \tDiracPropagator,
                and similar lattice variants\\
  \hline
  \end{tabular}
\end{flushleft}


\paragraph{Right multiplication by gauge matrix}

\begin{flushleft}
  \begin{tabular}{|l|l|}
  \hline
  Syntax      & {\it Type} \verb|operator*|({\tt const {\it Type}\& a, const {\it Type1}\& b})\\
  \hline
  Meaning     & $r = a * b$ \\
  \hline
  {\it Type1} & \tColorMatrix, \tLatticeColorMatrix\\
  \itt        & \tColorMatrix, \tSpinMatrix, \tDiracPropagator,
                and similar lattice variants\\
  \hline
  \end{tabular}
\end{flushleft}

\subsection{Boolean and Bit Operations}
\paragraph{Comparisons}

\begin{flushleft}
  \begin{tabular}{|l|l|}
  \hline 
  Syntax      & \protoBoolBinaryQual{op} \\
  \hline
  Meaning     & $r = a \op b$ or $r = \op(a,b)$ \\
  \hline
  $\op$       & \verb|<|, \verb|>|, \verb|!=|, \verb|<=|, \verb|>=|, \verb|==| \\
  \hline
  {\it Type1} & \tInt, \tReal, \tRealD, and similar lattice variants \\
  {\it Type2} & \tBoolean\ or \tLatticeBoolean\ (result is lattice if any arg is lattice)\\
  \hline
  \end{tabular}
\end{flushleft}

\paragraph{Elementary binary operations on integers}

\begin{flushleft}
  \begin{tabular}{|l|l|}
  \hline 
  Syntax      & \protoBoolBinaryQual{op} \\
  \hline
  Meaning     & $r = a \op b$ or $r = \op(a,b)$\\
  \hline
  $\op$       & \verb|<<|, \verb|>>|, \verb|&| (and), {\tt |} (or), \verb|^| (xor), 
                {\tt mod, max, min} \\
  \hline
  \end{tabular}
\end{flushleft}

\paragraph{Elementary binary operations on reals}

\begin{flushleft}
  \begin{tabular}{|l|l|}
  \hline
  Syntax      & \protoBinaryQual{op} \\
  \hline
  Meaning     & $r = a \op b$ or $r = \op(a,b)$ \\
  \hline
  $\op$       & {\tt mod, max, min} \\
  \hline
  \itt        & \tReal, \tRealD, and similar lattice variants \\
  \hline
  \end{tabular}
\end{flushleft}

\paragraph{Boolean Operations}

\begin{flushleft}
  \begin{tabular}{|l|l|}
  \hline
  Syntax      & \protoBBinaryQual{op} \\
  \hline
  Meaning     & $r = a \op b$\\
  \hline
  $\op$       & {\tt |} (or), \verb|&| (and), \verb|^| (xor) \\
  \hline
  \itt        & \tBoolean, \tLatticeBoolean\\
  \hline
  \end{tabular}
\end{flushleft}

\begin{flushleft}
  \begin{tabular}{|l|l|}
  \hline
  Syntax      & \protoUnaryQual{op} \\
  \hline
  Meaning     & $r = \not a$\\
  \hline
  $\op$       & \verb|!| (not) \\
  \hline
  \itt        & \tBoolean, \tLatticeBoolean\\
  \hline
  \end{tabular}
\end{flushleft}

\paragraph{Copymask}

\begin{flushleft}
  \begin{tabular}{|l|l|}
  \hline
  Syntax      & {\tt void} \verb|copymask|({\tt const {\it Type2}\&  r, const {\it Type1}\& a, const {\it Type1}\& b})\\
  \hline
  Meaning     & $r = b$ if $a$ is true\\
  \hline
  \itt        & \allNumericTypes \\
  \hline
  \end{tabular}
\end{flushleft}

\subsection{Reductions}

Global reductions sum over all lattice sites in the subset specified by 
the left hand side of the assignment.

\paragraph{Norms}

\begin{flushleft}
  \begin{tabular}{|l|l|}
  \hline
  Syntax      & {\tt \tReal} \verb|norm2|({\tt {\it Type}\& a})\\
  \hline
  Meaning     & $r = \sum |a|^2$ \\
  \hline
  \itt        & \allNumericTypes \\
  \hline
  \end{tabular}
\end{flushleft}

\paragraph{Inner products}

\begin{flushleft}
  \begin{tabular}{|l|l|}
  \hline
  Syntax      & {\tt \tComplex} \verb|innerProduct|({\tt {\it Type}\& a, const {\it Type}\& b})\\
  \hline
  Meaning     & $r = \sum a \cdot b$ \\
  \hline
  \itt        & \allNumericTypes \\
  \hline
  \end{tabular}
\end{flushleft}


\paragraph{Global sums}

\begin{flushleft}
  \begin{tabular}{|l|l|}
  \hline
  Syntax      & {\it Type} \verb|sum|({\tt const Lattice{\it Type}\& a})\\
  \hline
  Meaning     & $r = \sum a$\\
  \hline
  \itt     & All numeric non-lattice scalar types \\
  \hline
  \end{tabular}
\end{flushleft}


\subsection{Fills}

\paragraph{Coordinate function fills}

\begin{flushleft}
  \begin{tabular}{|l|l|}
  \hline
  Syntax      & {\tt \tLatticeInt} \verb|Layout::latticeCoordinate|({\tt int d})\\
  \hline
  Meaning     & $r = f(d)$ for direction \verb|d|.\\
  \hline
  Purpose     & Return the lattice coordinates in direction \verb|d|\\
  \hline
  \end{tabular}
\end{flushleft}
%
The call \verb|Layout::latticeCoordinate(d)| returns an integer lattice field with a
value on each site equal to the integer value of the \verb|d|th
space-time coordinate on that site.

\paragraph{Constant fills}

\begin{flushleft}
  \begin{tabular}{|l|l|}
  \hline
  Syntax      & {\tt Lattice{\it Type}} \verb|operator=|({\tt Lattice{\it Type}\& r, const {\it Type}\& a})\\
  \hline
  Meaning     & $r = a$ for all sites\\
  \hline
  \itt        & All non-lattice objects \\
  \hline
  Example     & {\tt \tReal{} a = 2.0;}\\
              & {\tt \tLatticeReal{} r = a;}\\
  \hline
  \end{tabular}
\end{flushleft}

Constant (or lattice global) fills are always defined for lattice scalar objects broadcasting to
all lattice sites. These are broadcasts of a lattice scalar type to a conforming lattice type.

NOTE, one can not fill a \tLatticeColorVector{} with a \tReal.

\begin{flushleft}
  \begin{tabular}{|l|l|}
  \hline
  Syntax      & {\tt Lattice{\it Type}} \verb|operator=|({\tt Lattice{\it Type}\& r, const {\it Type}\& a})\\
  \hline
  Meaning     & $r = \diag(a,a,\ldots{})$ (constant $a$)\\
  \hline
  \itt        & \tComplex, \tColorMatrix, \tSpinMatrix \\
  \hline
  Example     & {\tt \tReal{} a = 2.0;}\\
              & {\tt \tLatticeColorMatrix{} r = a;}\\
  \hline
  \end{tabular}
\end{flushleft}

Only sets the diagonal part of a field to a constant \verb|a| times
the identity.

This fill can only be used on primitive types that are scalars or
matrices. E.g., it can not be used for a {\it vector} field since
there is no meaning of diagonal.  NOTE, a zero cannot be distinguished
from a constant like 1. To initialize to zero the \verb|zero| argument
must be used.

\paragraph{Zero fills}

\begin{flushleft}
  \begin{tabular}{|l|l|}
  \hline
  Syntax      & {\it Type} \verb|operator=|({\tt {\it Type}\& r, const Zero\& zero})\\
  \hline
  Meaning     & $r = 0$\\
  \hline
  \itt        & \allNumericTypes \\
  \hline
  Example     & {\tt \tLatticeDiracFermion{} r = zero;}\\
  \hline
  \end{tabular}
\end{flushleft}

This is the only way to fill a vector field with a constant (like zero).

\paragraph{Uniform random number fills}

\begin{flushleft}
  \begin{tabular}{|l|l|}
  \hline
  Syntax      & {\tt void} \verb|random|({\tt {\it Type}\& r}) \\
  \hline
  Meaning     & $r$ random, uniform on $[0,1]$ \\
  \hline
  \itt        & \allFloatTypes \\
  \hline
  \end{tabular}
\end{flushleft}

\paragraph{Gaussian random number fills}

\begin{flushleft}
  \begin{tabular}{|l|l|}
  \hline
  Syntax      & {\tt void} \verb|gaussian|({\tt {\it Type}\& r}) \\
  \hline
  Meaning     & $r$ normal Gaussian \\
  \hline
  \itt        & \allFloatTypes \\
  \hline
  \end{tabular}
\end{flushleft}

\paragraph{Seeding the random number generator}

\begin{flushleft}
  \begin{tabular}{|l|l|}
  \hline
  Syntax      & {\tt void} \verb|RNG::setrn|({\tt const \tRandomState\& a}) \\
  \hline
  Meaning     & Initialize the random number generator with seed state \verb|a|\\
  \hline
  \end{tabular}
\end{flushleft}

For details see the discussion of the corresponding scalar function
{\tt random.h}.

\paragraph{Extracting the random number generator seed}

\begin{flushleft}
  \begin{tabular}{|l|l|}
  \hline
  Syntax      & {\tt void} \verb|RNG::savern|({\tt \tRandomState\& r}) \\
  \hline
  Meaning     & Extract the random number generator into seed state \verb|r|\\
  \hline
  \end{tabular}
\end{flushleft}

For details see the discussion of the corresponding scalar function
{\tt random.h}.

\end{document}

